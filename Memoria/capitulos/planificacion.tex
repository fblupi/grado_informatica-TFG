\chapter{Planificación}

En este capítulo comentaré la planificación inicial de tiempo en la que se llevará a cabo este TFG y la estimación de horas para cada tarea.

\section{Fechas y aclaraciones}

La primera reunión con mi tutor fue el día \textbf{11 de Noviembre}, así que se puede dar esa fecha como fecha de inicio. La fecha de entrega, en este momento en el que se realiza la planificación, es desconocida. Pero espero tener el software terminado para la última semana de Mayo o primera de Junio, por lo que he puesto como fecha de fin el \textbf{9 de Junio}.

Al llevarse a cabo un desarrollo evolutivo incremental, no están concretados todos los requisitos que satisfará el software, por lo que solo están definidos los requisitos iniciales, aunque si se han planificado horas para las posteriores mejoras. Estos requisitos se descompondrán en tareas una vez se definan y se irán acoplando al espacio temporal reservado.

Al hacerse esta planificación tras la segunda reunión, todas las tareas tanto de la primera como de la segunda no tienen el número de horas estimadas, sino las empleadas realmente.

La estimación para cada tarea ha sido difícil de asignar por ser la primera vez que voy a trabajar con VTK y Qt, pero espero que la experiencia que he adquirido en las prácticas que he realizado en las distintas asignaturas que he tenido durante estos cuatro años me ayude y logre hacer una buena planificación.

\section{Planificación}

He troceado el calendario con un \textit{sprint} de reunión en reunión (cada dos semanas) y el resultado ha sido el siguiente:

\begin{itemize}
	\item \textbf{Reunión 1} (13/11/15 - 26/11/15)
	\begin{itemize}
		\item Instalación de entorno de desarrollo (12 horas)
		\item Aprender VTK (4 horas)
		\item Aprender Qt (3 horas)
		\item Aprender estructura DICOM (3 horas)
	\end{itemize}
	
	\item \textbf{Reunión 2} (27/11/15 - 10/12/15)
	\begin{itemize}
		\item Especificación de requisitos (6 horas)
		\item Planificación (2 horas)
		\item Visualización de volumen básica (10 horas)
	\end{itemize}
	
	\item \textbf{Reunión 3} (11/12/15 - 7/1/16)
	\begin{itemize}
		\item Interacción con la cámara (8 horas)
		\item Estudiar función de transferencia (12 horas)
		\item Implementar función de transferencia (12 horas)
		\item Escribir en la memoria (4 horas)
	\end{itemize}
	
	\item \textbf{Reunión 4} (8/1/16 - 21/1/16)
	\begin{itemize}
		\item Modificar función de transferencia (15 horas)
		\item Visualizar plano (5 horas)
	\end{itemize}
	
	\item \textbf{Reunión 5} (22/1/16 - 4/2/16)
	\begin{itemize}
		\item Modificar plano dando puntos (5 horas)
		\item Modificar plano interactuando con éste (15 horas)
	\end{itemize}
	
	\item \textbf{Reunión 6} (5/2/16 - 18/2/16)
	\begin{itemize}
		\item Generar corte con el plano (8 horas)
		\item Visualizar corte generado (12 horas)
	\end{itemize}
	
	\item \textbf{Reunión 7} (19/2/16 - 3/3/16)
	\begin{itemize}
		\item Interactuar con cortes generados (6 horas)
		\item Guardar imágenes (8 horas)
		\item Testeos intensivos (4 horas)
		\item Corrección de \textit{bugs} (6 horas)
		\item Escribir en la memoria (4 horas)
	\end{itemize}
	
	\item \textbf{Reunión 8} (4/3/16 - 17/3/16)
	\begin{itemize}
		\item Mejora \#1 (18 horas)
		\item Escribir en la memoria (2 horas)
	\end{itemize}
	
	\item \textbf{Reunión 9} (18/3/16 - 31/3/16)
	\begin{itemize}
		\item Mejora \#2 (18 horas)
		\item Escribir en la memoria (2 horas)
	\end{itemize}
	
	\item \textbf{Reunión 10} (1/4/16 - 14/4/16)
	\begin{itemize}
		\item Mejora \#3 (18 horas)
		\item Escribir en la memoria (2 horas)
	\end{itemize}
	
	\item \textbf{Reunión 11} (15/4/16 - 28/4/16)
	\begin{itemize}
		\item Mejora \#4 (18 horas)
		\item Escribir en la memoria (2 horas)
	\end{itemize}
	
	\item \textbf{Reunión 12} (29/4/16 - 12/5/16)
	\begin{itemize}
		\item Mejora \#5 (18 horas)
		\item Escribir en la memoria (2 horas)
	\end{itemize}
	
	\item \textbf{Reunión 13} (13/5/16 - 26/5/16)
	\begin{itemize}
		\item Mejora \#6 (18 horas)
		\item Escribir en la memoria (2 horas)
	\end{itemize}
	
	\item \textbf{Reunión 14} (27/5/16 - 9/6/16)
	\begin{itemize}
		\item Revisar y terminar la memoria (6 horas)
		\item Preparar la exposición (10 horas)
	\end{itemize}
\end{itemize}

En total se estima que se realicen unas \textbf{300 horas}. Se llevará a cabo un recuento de horas realizadas para ver, finalmente, cuántas se han necesitado.

\section{Ciclo de desarrollo}

Se seguirá un \textbf{desarrollo evolutivo} basado en un \textbf{prototipo} totalmente funcional. 

El alcance total del proyecto no está definido, tan solo unos requisitos iniciales y una vez sean cumplidos se irá añadiendo nueva funcionalidad al software.

En primer lugar se realizará un estudio del problema y un documento de especificación de requisitos. Posteriormente se hará uso de \textbf{historias de usuario} para detectar y registrar la funcionalidad del software.

Se documentará con un diagrama arquitectónico, diagramas de clases y de secuencia. Asimismo se mantendrá el código comentado y documentado.