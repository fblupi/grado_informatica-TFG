\documentclass[a4paper,11pt]{book}
\usepackage{listings}
\usepackage{xspace}
\usepackage{url}
\usepackage[utf8]{inputenc}
\usepackage[spanish]{babel}

%\usepackage[style=list, number=none]{glossary} %
%\usepackage{titlesec}
%\usepackage{pailatino}

\decimalpoint
\usepackage{dcolumn}
\newcolumntype{.}{D{.}{\esperiod}{-1}}
\makeatletter
\addto\shorthandsspanish{\let\esperiod\es@period@code}
\makeatother


%\usepackage[chapter]{algorithm}
\RequirePackage{verbatim}
%\RequirePackage[Glenn]{fncychap}
\usepackage{fancyhdr}
\usepackage{graphics, graphicx, float}
\usepackage{afterpage}

\usepackage{longtable}

\usepackage[pdfborder={000}]{hyperref} %referencia

% ********************************************************************
% Re-usable information
% ********************************************************************
\newcommand{\myTitle}{3DCurator\xspace}
\newcommand{\mySubtitle}{Un visor 3D de TCs de esculturas\xspace}
\newcommand{\myEnglishSubtitle}{A 3D Viewer for CTs of Polychromed Wood Sculptures\xspace}
\newcommand{\myDegree}{Grado en Ingeniería Informática\xspace}
\newcommand{\myName}{Francisco Javier Bolívar Lupiáñez\xspace}
\newcommand{\myProf}{Francisco Javier Melero Rus\xspace}
\newcommand{\myFaculty}{Escuela Técnica Superior de Ingenierías Informática y de Telecomunicación\xspace}
\newcommand{\myFacultyShort}{E.T.S. de Ingenierías Informática y de Telecomunicación\xspace}
\newcommand{\myDepartment}{Departamento de Lenguajes y Sistemas Informáticos\xspace}
\newcommand{\myUni}{\protect{Universidad de Granada}\xspace}
\newcommand{\myLocation}{Granada\xspace}
\newcommand{\myTime}{\today\xspace}
\newcommand{\myVersion}{Version 0.1\xspace}


\hypersetup{
pdfauthor = {\myName (fblupi@correo.ugr.es)},
pdftitle = {\myTitle},
pdfsubject = {},
pdfkeywords = {3DCurator, informática gráfica, renderizado de volúmenes, tomografía computarizada, escultura policromada de madera, restauración, conservador de arte},
pdfproducer = {pdflatex}
}

%\hyphenation{}


%\usepackage{doxygen/doxygen}
%\usepackage{pdfpages}
\usepackage{url}
\usepackage{colortbl,longtable}
\usepackage[stable]{footmisc}
%\usepackage{index}

%\makeindex
%\usepackage[style=long, cols=2,border=plain,toc=true,number=none]{glossary}
% \makeglossary

% Definición de comandos que me son tiles:
%\renewcommand{\indexname}{Índice alfabético}
%\renewcommand{\glossaryname}{Glosario}

\pagestyle{fancy}
\fancyhf{}
\fancyhead[LO]{\leftmark}
\fancyhead[RE]{\rightmark}
\fancyhead[RO,LE]{\textbf{\thepage}}
\renewcommand{\chaptermark}[1]{\markboth{\textbf{#1}}{}}
\renewcommand{\sectionmark}[1]{\markright{\textbf{\thesection. #1}}}

\setlength{\headheight}{1.5\headheight}

\newcommand*\justify{
	\fontdimen2\font=0.4em
	\fontdimen3\font=0.2em
	\fontdimen4\font=0.1em
	\fontdimen7\font=0.1em
	\hyphenchar\font=`\-
}
\newcommand{\HRule}{\rule{\linewidth}{0.5mm}}
%Definimos los tipos teorema, ejemplo y definición podremos usar estos tipos
%simplemente poniendo \begin{teorema} \end{teorema} ...
\newtheorem{teorema}{Teorema}[chapter]
\newtheorem{ejemplo}{Ejemplo}[chapter]
\newtheorem{definicion}{Definición}[chapter]

\definecolor{gray97}{gray}{.97}
\definecolor{gray75}{gray}{.75}
\definecolor{gray45}{gray}{.45}
\definecolor{gray30}{gray}{.94}

\lstset{ frame=Ltb,
     framerule=0.5pt,
     aboveskip=0.5cm,
     framextopmargin=3pt,
     framexbottommargin=3pt,
     framexleftmargin=0.1cm,
     framesep=0pt,
     rulesep=.4pt,
     backgroundcolor=\color{gray97},
     rulesepcolor=\color{black},
     %
     stringstyle=\ttfamily,
     showstringspaces = false,
     basicstyle=\scriptsize\ttfamily,
     commentstyle=\color{gray45},
     keywordstyle=\bfseries,
     %
     numbers=left,
     numbersep=6pt,
     numberstyle=\tiny,
     numberfirstline = false,
     breaklines=true,
   }
 
% minimizar fragmentado de listados
\lstnewenvironment{listing}[1][]
   {\lstset{#1}\pagebreak[0]}{\pagebreak[0]}

\lstdefinestyle{C} {
	basicstyle=\scriptsize,
	frame=single,
	language=C,
	numbers=left
}

\lstdefinestyle{C++} {
	basicstyle=\small,
	frame=single,
	backgroundcolor=\color{gray30},
	language=C++,
	numbers=left
}

\lstdefinestyle{Consola} {
   	basicstyle=\scriptsize\bf\ttfamily,
    backgroundcolor=\color{gray30},
    frame=single,
    language=shell,
    numbers=none
}

\lstdefinestyle{XML} {
	basicstyle=\scriptsize,
	frame=single,
	language=XML,
	numbers=left
}

\newcommand{\bigrule}{\titlerule[0.5mm]}


%Para conseguir que en las páginas en blanco no ponga cabecerass
\makeatletter
\def\clearpage{%
  \ifvmode
    \ifnum \@dbltopnum =\m@ne
      \ifdim \pagetotal <\topskip
        \hbox{}
      \fi
    \fi
  \fi
  \newpage
  \thispagestyle{empty}
  \write\m@ne{}
  \vbox{}
  \penalty -\@Mi
}
\makeatother

\usepackage{pdfpages}
\begin{document}
\input{portada/portada}
\chapter*{}

\begin{titlepage}
 
 
\setlength{\centeroffset}{-0.5\oddsidemargin}
\addtolength{\centeroffset}{0.5\evensidemargin}
\thispagestyle{empty}

\noindent\hspace*{\centeroffset}
\begin{minipage}{\textwidth}
\centering
\vspace{3.3cm}
%\includegraphics{imagenes/logo.png} 
%\vspace{0.5cm}

{\Huge\bfseries \myTitle \\}
\noindent\rule[-1ex]{\textwidth}{3pt}\\[3.5ex]
{\large\bfseries \mySubtitle \\[4cm]}
\end{minipage}

\vspace{2.5cm}

\noindent\hspace*{\centeroffset}
\begin{minipage}{\textwidth}
\centering
\textbf{Autor}\\ {\myName}\\[2.5ex]
\textbf{Director}\\ {\myProf}\\[2cm]
\end{minipage}

\vspace{\stretch{2}}

\end{titlepage}




\cleardoublepage
\thispagestyle{empty}

\begin{center}
{\large\bfseries \myTitle: \mySubtitle}\\
\end{center}

\begin{center}
\myName \\
\end{center}

\vspace{0.7cm}
\noindent{\textbf{Palabras clave}: informática gráfica, renderizado de volúmenes, tomografía computarizada, escultura policromada de madera, restauración, conservador de arte}\\

\vspace{0.7cm}
\noindent{\textbf{Resumen}}\\

El objetivo principal de este proyecto es desarrollar un software diseñado para la documentación de datos volumétricos de esculturas policromadas de madera obtenidos mediante una Tomografía Computarizada (TC). El uso de esta técnica no destructiva permite a los expertos examinar el estado de la escultura de una manera más exhaustiva que usando la radiografía tradicional, la cual presenta limitaciones como la superposición de planos. La mayor parte de herramientas de renderizado volumétrico son muy caras y están orientadas a datos médicos, lo que las hace poco útiles para los restauradores debido a su precio y sus opciones de renderizado enfocadas al cuerpo humano. El software desarrollado permite cargar imágenes DICOM y realizar operaciones de renderizado y documentación como tomar medidas o extraer cortes en cualquier orientación.

\thispagestyle{empty}

\cleardoublepage

\begin{center}
	{\large\bfseries \myTitle: \myEnglishSubtitle}\\
\end{center}

\begin{center}
	\myName \\
\end{center}

\vspace{0.7cm}
\noindent{\textbf{Keywords}: Computer Graphics, Volume Rendering, Computed Tomography, Polychromed Wood Sculpture, Restoration, Art Curator}\\

\vspace{0.7cm}
\noindent{\textbf{Abstract}}\\

The main objective of this project is to develop a new software designed for the 3D documentation of wood sculptures from Computer Tomography (CT) datasets. This nondestructive technique allows experts to examine the artwork in a more confident and exhaustive manner than using the traditional X-ray plate which presents some limitations such as overlapping planes. Most of current available volume rendering tools are very expensive and orientated towards medical datasets, which makes unusable by art curators due to its price and its human-focused rendering options. The developed software loads DICOM files and performs rendering operations and several documentation actions, such as taking measurements or extracting slices at any orientation.


\chapter*{}
\thispagestyle{empty}

\noindent\rule[-1ex]{\textwidth}{2pt}\\[4.5ex]

Yo, \textbf{\myName}, alumno de la titulación \myDegree de la \textbf{\myFaculty}, con DNI 75926571Y, autorizo la ubicación de la siguiente copia de mi Trabajo Fin de Grado en la biblioteca del centro para que pueda ser consultada por las personas que lo deseen.

\vspace{6cm}

\noindent Fdo: \myName

\vspace{2cm}

\begin{flushright}
\myLocation a \myTime.
\end{flushright}


\chapter*{}
\thispagestyle{empty}

\noindent\rule[-1ex]{\textwidth}{2pt}\\[4.5ex]

D. \textbf{\myProf}, Profesor del Área de XXXX del \myDepartment de la \myUni.

\vspace{0.5cm}

\textbf{Informa:}

\vspace{0.5cm}

Que el presente trabajo, titulado \textit{\textbf{\myTitle, \mySubtitle}}, ha sido realizado bajo su supervisión por \textbf{\myName}, y autorizamos la defensa de dicho trabajo ante el tribunal que corresponda.

\vspace{0.5cm}

Y para que conste, expiden y firman el presente informe en \myLocation a \myTime.

\vspace{1cm}

\textbf{El director:}

\vspace{5cm}

\noindent \textbf{\myProf}

\chapter*{Agradecimientos}
\thispagestyle{empty}

\vspace{1cm}

No podría empezar de otra manera que dedicando unas palabras de agradecimiento a mis padres. Han luchado por mi educación y hoy estarán orgullosos de mi tanto como yo de ellos.
\\

Gracias a Javier Melero, mi tutor, por haberme guiado y motivado durante todo el curso. Sin él esto no sería posible.
\\

Gracias a todos los familiares y amigos con los que me he desahogado cuando algo no salía y me han escuchado pese a, muchas veces, no tener ni idea de lo que les estaba contando.
\\

Gracias también a mis compañeros de clase que no solo me han escuchado sino que también me han propuesto soluciones a problemas que surgían.
\\

Y como no, a \textit{Los Hijos de Vuctir} que me han acompañado durante estos cuatro años y han hecho más llevaderos los días de clase. No solo me llevo conocimientos de la universidad, también me llevo grandes amigos que pueden contar conmigo para lo que sea.
\frontmatter
\tableofcontents
\listoffigures
\listoftables
%
\mainmatter
\setlength{\parskip}{5pt}

\chapter{Introducción}
El objetivo de este proyecto es construir un software con el que poder visualizar e interactuar con los datos DICOM obtenidos al someter a una escultura a una Tomografía Axial Computerizada (TAC). 

Para ello se hará uso de VTK, que proporciona una serie de librerías en C++ para facilitar operaciones sobre datos DICOM, y de Qt, para la Interfaz Gráfica de Usuario (GUI).

Antes de empezar con el proyecto en sí, se definirán conceptos como DICOM o TAC que se usarán a lo largo de éste y conviene saber lo que son, así como las distintas herramientas que se utilizarán.

\section{DICOM}
DICOM (\textit{Digital Imaging and Comunication in Medicine}) es el estándar internacional para manejar, visualizar, almacenar, imprimir y transmitir imágenes de pruebas médicas (ISO12052) \cite{about_dicom}.

Pese a que su uso está mayoritariamente extendido en el campo de la medicina, se puede usar en otros, como el de la restauración de bienes culturales, como es el caso de este proyecto.

En un archivo DICOM hay almacenado, además de metadatos, una imagen \cite{dicom_classes_vtk} (Figura~\ref{fig:prostate_dicom}).

\begin{figure}[H]
	\centering
	\includegraphics[width=10cm]{imagenes/prostate_dicom}
	\caption{Imagen DICOM de una próstata visualizada con un programa diseñado para visualizar archivos DICOM.}
	\label{fig:prostate_dicom}
\end{figure}

Cuando se realiza un TAC se obtiene una serie de imágenes (Figura~\ref{fig:brain_dicom_serie}) de rebanadas del objeto al que se realiza el escáner. Éstas imágenes son archivos DICOM, y con todas ellas se puede llegar a construir un modelo volumétrico.

\begin{figure}[H]
	\centering
	\includegraphics[width=10cm]{imagenes/brain_dicom_serie}
	\caption{Serie de imágenes DICOM extraídas de un TAC realizada a un cerebro.}
	\label{fig:brain_dicom_serie}
\end{figure}
\chapter{Especificación de requisitos}

Este capítulo es una Especificación de Requisitos Software (ERS) para el software que se va a realizar siguiendo las directrices dadas por el estándar IEEE830 \cite{iee830}.

\section{Introducción}

\subsection{Propósito}

Este capítulo de especificación de requisitos tiene como objetivo definir las especificaciones funcionales y no funcionales para el desarrollo de un software que permitirá visualizar e interactuar con los datos DICOM obtenidos al someter a una escultura a un TAC. Éste será utilizado principalmente por restauradores.

\subsection{Ámbito del sistema}

En la actualidad los datos DICOM obtenidos tras un TAC se utilizan, principalmente, en el campo donde surgieron, la medicina. No obstante, esto no significa que solo se pueda aplicar ahí. Con este software, llamadao \myTitle, se tratará de trasladar esta técnica al campo de la restauración de bienes culturales y poder visualizar e interactuar con los datos DICOM obtenidos con esculturas.

\subsection{Definiciones, acrónimos y abreviaturas}

\begin{itemize}
	\item \textbf{ERS}: Especificación de Requisitos Software
	\item \textbf{Lele}: Lolo
\end{itemize}

\subsection{Visión general del documento}

Este capítulo consta de tres secciones:
\begin{itemize}
	\item En la primera sección se realiza una introducción a éste y se proporciona una visión general de la ERS.
	\item En la segunda sección se realiza una descripción general a alto nivel del software, describiendo los factores que afectan al producto y a sus requisitos y con el objetivo de conocer las principales funcionalidades de éste.
	\item En la tercera sección se definen detalladamente los requisitos que deberá satisfacer el software.
\end{itemize}

\section{Descripción general}

\subsection{Perspectiva del producto}

\subsection{Funciones del producto}

\subsection{Características de los usuarios}

\subsection{Restricciones}

\subsection{Suposiciones y dependencias}

\subsection{Requisitos futuros}

\section{Requisitos específicos}

\subsection{Interfaces externas}

\subsection{Funciones}

\subsection{Requisitos de rendimiento}

\subsection{Restricciones de diseño}

\subsection{Atributos del sistema}

\subsection{Otros requisitos}

\section{Apéndices}
\chapter{Planificación}

En este capítulo comentaré la planificación inicial de tiempo en la que se llevará a cabo este TFG y la estimación de horas para cada tarea.

\section{Fechas y aclaraciones}

La primera reunión con mi tutor fue el día \textbf{11 de Noviembre}, así que se puede dar esa fecha como fecha de inicio. La fecha de entrega, en este momento en el que se realiza la planificación, es desconocida. Pero espero tener el software terminado para la última semana de Mayo o primera de Junio, por lo que he puesto como fecha de fin el \textbf{9 de Junio}.

Al llevarse a cabo un desarrollo evolutivo incremental, no están concretados todos los requisitos que satisfará el software, por lo que solo están definidos los requisitos iniciales, aunque si se han planificado horas para las posteriores mejoras. Estos requisitos se descompondrán en tareas una vez se definan y se irán acoplando al espacio temporal reservado.

Al hacerse esta planificación tras la segunda reunión, todas las tareas tanto de la primera como de la segunda no tienen el número de horas estimadas, sino las empleadas realmente.

La estimación para cada tarea ha sido difícil de asignar por ser la primera vez que voy a trabajar con VTK y Qt, pero espero que la experiencia que he adquirido en las prácticas que he realizado en las distintas asignaturas que he tenido durante estos cuatro años me ayude y logre hacer una buena planificación.

\section{Planificación}

He troceado el calendario con un \textit{sprint} de reunión en reunión (cada dos semanas) y el resultado ha sido el siguiente:

\begin{itemize}
	\item \textbf{Reunión 1} (13/11/15 - 26/11/15)
	\begin{itemize}
		\item Instalación de entorno de desarrollo (12 horas)
		\item Aprender VTK (4 horas)
		\item Aprender Qt (3 horas)
		\item Aprender estructura DICOM (3 horas)
	\end{itemize}
	
	\item \textbf{Reunión 2} (27/11/15 - 10/12/15)
	\begin{itemize}
		\item Especificación de requisitos (6 horas)
		\item Planificación (2 horas)
		\item Visualización de volumen básica (10 horas)
	\end{itemize}
	
	\item \textbf{Reunión 3} (11/12/15 - 7/1/16)
	\begin{itemize}
		\item Interacción con la cámara (8 horas)
		\item Estudiar función de transferencia (12 horas)
		\item Implementar función de transferencia (12 horas)
		\item Escribir en la memoria (4 horas)
	\end{itemize}
	
	\item \textbf{Reunión 4} (8/1/16 - 21/1/16)
	\begin{itemize}
		\item Modificar función de transferencia (15 horas)
		\item Visualizar plano (5 horas)
	\end{itemize}
	
	\item \textbf{Reunión 5} (22/1/16 - 4/2/16)
	\begin{itemize}
		\item Modificar plano dando puntos (5 horas)
		\item Modificar plano interactuando con éste (15 horas)
	\end{itemize}
	
	\item \textbf{Reunión 6} (5/2/16 - 18/2/16)
	\begin{itemize}
		\item Generar corte con el plano (8 horas)
		\item Visualizar corte generado (12 horas)
	\end{itemize}
	
	\item \textbf{Reunión 7} (19/2/16 - 3/3/16)
	\begin{itemize}
		\item Interactuar con cortes generados (6 horas)
		\item Guardar imágenes (8 horas)
		\item Testeos intensivos (4 horas)
		\item Corrección de \textit{bugs} (6 horas)
		\item Escribir en la memoria (4 horas)
	\end{itemize}
	
	\item \textbf{Reunión 8} (4/3/16 - 17/3/16)
	\begin{itemize}
		\item Mejora \#1 (18 horas)
		\item Escribir en la memoria (2 horas)
	\end{itemize}
	
	\item \textbf{Reunión 9} (18/3/16 - 31/3/16)
	\begin{itemize}
		\item Mejora \#2 (18 horas)
		\item Escribir en la memoria (2 horas)
	\end{itemize}
	
	\item \textbf{Reunión 10} (1/4/16 - 14/4/16)
	\begin{itemize}
		\item Mejora \#3 (18 horas)
		\item Escribir en la memoria (2 horas)
	\end{itemize}
	
	\item \textbf{Reunión 11} (15/4/16 - 28/4/16)
	\begin{itemize}
		\item Mejora \#4 (18 horas)
		\item Escribir en la memoria (2 horas)
	\end{itemize}
	
	\item \textbf{Reunión 12} (29/4/16 - 12/5/16)
	\begin{itemize}
		\item Mejora \#5 (18 horas)
		\item Escribir en la memoria (2 horas)
	\end{itemize}
	
	\item \textbf{Reunión 13} (13/5/16 - 26/5/16)
	\begin{itemize}
		\item Mejora \#6 (18 horas)
		\item Escribir en la memoria (2 horas)
	\end{itemize}
	
	\item \textbf{Reunión 14} (27/5/16 - 9/6/16)
	\begin{itemize}
		\item Revisar y terminar la memoria (6 horas)
		\item Preparar la exposición (10 horas)
	\end{itemize}
\end{itemize}

En total se estima que se realicen unas \textbf{300 horas}. Se llevará a cabo un recuento de horas realizadas para ver, finalmente, cuántas se han necesitado.

\section{Ciclo de desarrollo}

Se seguirá un \textbf{desarrollo evolutivo} basado en un \textbf{prototipo} totalmente funcional. 

El alcance total del proyecto no está definido, tan solo unos requisitos iniciales y una vez sean cumplidos se irá añadiendo nueva funcionalidad al software.

En primer lugar se realizará un estudio del problema y un documento de especificación de requisitos. Posteriormente se hará uso de \textbf{historias de usuario} para detectar y registrar la funcionalidad del software.

Se documentará con un diagrama arquitectónico, diagramas de clases y de secuencia. Asimismo se mantendrá el código comentado y documentado.
\chapter{Análisis}

\section{Técnicas de renderizado}

A la hora de renderizar un conjunto de datos volumétricos para obtener una imagen en 3D, se pueden utilizar distintas técnicas y VTK proporciona una serie de clases para su uso:
\begin{itemize}
	\item \textbf{\textit{Marching Cubes}}: Con este algoritmo se obtiene una malla poligonal de una isosuperficie a partir de un conjunto de datos volumétrico (Figura \ref{fig:marching_cubes_head}) \cite{marching_cubes}. Se puede usar en VTK con \texttt{vtkMarchingCubes}.
	\begin{figure}[H]
		\centering
		\includegraphics[width=6cm]{imagenes/marching_cubes_head}
		\caption{Cabeza extraída de 150 cortes obtenidos por una IRM usando \textit{marching cubes} (sobre 150.000 triángulos). Imagen extraída de \url{https://en.wikipedia.org/wiki/File:Marchingcubes-head.png}}
		\label{fig:marching_cubes_head}
	\end{figure}
	
	\item \textbf{Texturas2D}: Se utilizan planos de corte alineados a los ejes de coordenadas. Por lo que se tendría una serie de cortes sobre el plano sagital, otra sobre el coronal y otra sobre el axial. Se realiza una interpolación bilineal para obtener la imagen final (Figura \ref{fig:texturas2d} \cite{intro_medical_vtk_bioimage}). Se puede usar en VTK con \texttt{vtkVolumeTextureMapper}.
	\begin{figure}[H]
		\centering
		\includegraphics[width=10cm]{imagenes/texturas2d}
		\caption{Esquema del proceso de renderizado usando texturas 2D. Imagen extraída del apéndice B del libro \textit{An Introduction to Programming for Medical Image Analysis with the Visualization Toolkit }\cite{intro_medical_vtk_bioimage}}
		\label{fig:texturas2d}
	\end{figure}
	
	\item \textbf{Texturas3D}: Esta técnica es similar a la anterior, pero ahora los datos se cargan en una textura 3D y los cortes se dibujan paralelos a la dirección de vista. A diferencia de las texturas 2D, usa interpolación trilineal y no es necesario tener almacenado en memoria tres copias de los mismos datos (Figura \ref{fig:texturas3d}) \cite{intro_medical_vtk_bioimage}. Se puede usar en VTK con \texttt{vtkVolumeTextureMapper3D}.
	\begin{figure}[H]
		\centering
		\includegraphics[width=10cm]{imagenes/texturas3d}
		\caption{Esquema del proceso de renderizado usando texturas 3D. Imagen extraída del apéndice B del libro \textit{An Introduction to Programming for Medical Image Analysis with the Visualization Toolkit} \cite{intro_medical_vtk_bioimage}}
		\label{fig:texturas3d}
	\end{figure}
	
	\item \textbf{\textit{Volume Ray Casting}}: Es una técnica en el que para cada pixel de la imagen se lanza un rayo que atraviesa el volumen. Para cada voxel se obtiene su color y opacidad usando una función de transferencia. Cuando el rayo sale del volumen se calcula el color y opacidad del pixel como el acumulado por el rayo. Existe una versión de este algoritmo que hace uso de la GPU para acelerar ostensiblemente el tiempo de la operación (Figura \ref{fig:volume_ray_casting}) \cite{intro_medical_vtk_bioimage}. Se puede usar en VTK con \texttt{vtkFixedVolumeRayCastMapper}, \texttt{vtkVolumeRayCastMapper} (usan CPU), \texttt{vtkGPUVolumeRayCastMapper} (usa GPU) y \texttt{vtkSmartVolumeMapper} (según el contexto usa CPU o GPU).
	\begin{figure}[H]
		\centering
		\includegraphics[width=12.5cm]{imagenes/volume_ray_casting}
		\caption{Esquema del proceso de \textit{ray casting}. Imagen extraída de \url{https://en.wikipedia.org/wiki/File:Volume_ray_casting.png}}
		\label{fig:volume_ray_casting}
	\end{figure}
\end{itemize}

De entre todas estas técnicas, se podrían descartar rápidamente la de \textit{marching cubes}: pues tan solo trabaja con isosuperficies y la de texturas 2D: pues la opción de texturas 3D es más rápida y usa menos recursos.

Por tanto ya solo habría que elegir entre texturas 3D o \textit{ray casting}. Hasta hace unos años, VTK no proporcionaba un algoritmo de \textit{ray casting} que usase la GPU. Por tanto la opción habría sido sencilla, pero durante los últimos años han trabajado en esto haciendo del \textit{ray casting} la opción preferible.

\section{Volume Mapper}

Para poder visualizar un volumen con VTK mediante Direct Volume Rendering (DVR), necesitamos un \textit{Volume Mapper}. La librería nos ofrece varias alternativas:
\begin{itemize}
	\item \texttt{vtkAMRVolumeMapper}
	\item \texttt{vtkFixedVolumeRayCastMapper}
	\item \texttt{vtkGPUVolumeRayCastMapper}
	\item \texttt{vtkSmartVolumeMapper}
	\item \texttt{vtkVolumeRayCastMapper}
	\item \texttt{vtkVolumeTextureMapper}
	\item \texttt{vtkVolumeTextureMapper3D}
\end{itemize}

Entre esta lista tenemos algunos que utilizan o texturas o \textit{ray casting}, o la CPU o la GPU. Pero hay uno que es especial con respecto al resto. Se trata de \texttt{vtkSmartVolumeMapper}.

Este \textit{Volume Mapper} es una versión mejorada del \texttt{vtkGPUVolumeRayCast Mapper} por lo que utiliza la GPU (si el dispositivo cuenta con una) y la técnica de \textit{ray casting}. Además cuenta con nuevas características con respecto al resto, como el poder definir infinitos planos de corte para poder ver el interior del volumen \cite{smart_volume_mapper}.

Por tanto, el \textit{Volume Mapper} utilizado será el \texttt{vtkSmartVolumeMapper}.

\section{Función de transferencia}

La función de transferencia es la encargada de dar a un valor de intensidad las propiedades de color y opacidad que le corresponden para la visualización del volumen.

En VTK la función de transferencia forma parte de la clase \texttt{vtkVolume Property} \cite{vtk_example_medical4}. Para ello proporciona otras dos clases:
\begin{itemize}
	\item \texttt{vtkColorTransferFunction}: Para definir el color. Se enlaza a \texttt{vtk VolumeProperty} con el método \texttt{SetColor}.
	\item \texttt{vtkPiecewiseFunction}: Para definir la opacidad (tanto escalar como gradiente). La opacidad escalar se enlaza a \texttt{vtkVolumeProperty} con el método \texttt{SetScalarOpacity} y la gradiente con \texttt{SetGradientOpacity}.
\end{itemize}

Podemos, por tanto, diferenciar tres partes fundamentales en la función de transferencia, la encargada de dar la propiedad de color y las dos de dar la propiedad de opacidad. Ambas trabajan de forma independiente. Es decir, cuando se define un punto en una de ellas, no tiene por qué definirse en la otra.

\subsection{Color}

Para definir esta función (\texttt{vtkColorTransferFunction}), hay que agregar puntos para valores de intensidad a los que se les asignará un color. VTK se encargará de interpolar entre un punto y otro (Figura \ref{fig:color_tf}). 

Por defecto, cuando no hay ningún punto, a todos los valores de intensidad les corresponderá un color negro. De forma parecida se comporta cuando solo hay un punto pero en lugar de negro, les corresponderá el color del punto que se ha definido.

VTK permite trabajar tanto con HSV como con RGB y para añadir un punto hay que utilizar \texttt{AddHSVPoint} o \texttt{AddRGBPoint}. A estos métodos se les pasa un primer parámetro en coma flotante con el valor de intensidad donde se establecerá ese punto y otros tres con las distintas exponentes (\textit{hue}, \textit{saturation}, \textit{brightness} o \textit{red}, \textit{green}, \textit{blue}).

\begin{figure}[H]
	\centering
	\includegraphics[width=12.5cm]{imagenes/color_tf}
	\caption{Parte de color de la función de transferencia del \textit{preset} \textit{CT-WoodSculpture} creado para visualizar esculturas de madera policromadas. Dos puntos definen el color. Uno en -750 con un tono más oscuro y otro en -350 con un tono más claro. El estuco se pinta con un color gris claro y también viene definido por dos puntos: -200 y 2750. Finalmente, el metal se verá con un tono gris oscuro definido con un punto en 3000.}
	\label{fig:color_tf}
\end{figure}

\subsection{Opacidad}

El valor de opacidad se obtendría como el \textbf{producto de la opacidad escalar por la gradiente}. Si no se define alguna de las dos, se definiría como un valor constante de 1 para que solo se viese el resultado de la que sí está definida.

\subsubsection{Opacidad escalar}

Con el color no bastaría, pues si comprobásemos ahora añadiéndole tan solo el \texttt{vtkColorTransferFunction} al \texttt{vtkVolumeProperty} observaríamos que no se pinta nada en pantalla. Esto es porque por defecto, al no tener ningún punto la función de opacidad (\texttt{vtkPiecewiseFunction}) es una constante con valor 0 (transparente). 

Para definir esta función se trabaja de forma parecida a como se hace con el color, añadiendo puntos. El método que hay que utilizar es \texttt{AddPoint} al que se le pasan dos parámetros en coma flotante. El primero con el valor de intensidad y el segundo con la opacidad en ese punto. Para obtener los valores en puntos intermedios, se interpola entre los dos puntos en los que está. De forma que si para el valor de intensidad 100 hemos definido una opacidad de 0.5 y para el de 200 1, al valor de intensidad 150 le corresponderá 0.75.

Combinando color y opacidad escalar podemos obtener una función de transferencia para visualizar nuestro volumen (Figura \ref{fig:opacity_tf}), pero para obtener mejores resultados, habrá que utilizar la opacidad gradiente. 

\begin{figure}[H]
	\centering
	\includegraphics[width=12.5cm]{imagenes/opacity_tf}
	\caption{Parte de opacidad escalar de la función de transferencia del \textit{preset} \textit{CT-WoodSculpture} creado para visualizar esculturas de madera policromadas. Se pueden observar tres regiones. La primera corresponde a la madera, la segunda al estuco y la última al metal}
	\label{fig:opacity_tf}
\end{figure}

\subsubsection{Opacidad gradiente}

La opacidad gradiente utiliza el vector gradiente para dar el valor de opacidad. Con éste se puede conseguir \textbf{dar un mayor valor a regiones de los bordes, así como menor a regiones planas} es decir, donde no varía el valor de intensidad de sus vecinos de alrededor.

El gradiente se mide como la cantidad que varía la intensidad en una unidad de distancia. Este cálculo del gradiente lo realiza VTK cuando genera el volumen de forma transparente sin que haya que añadir nada al código.

Para poder definir la función de la opacidad gradiente, al igual que con las demás, hay que añadir puntos con la misma función que se usaba con la opacidad escalar (\texttt{AddPoint}). 

\begin{figure}[H]
	\centering
	\includegraphics[width=12.5cm]{imagenes/gradient_tf}
	\caption{Parte de opacidad gradiente de la función de transferencia del \textit{preset} \textit{CT-WoodSculpture} creado para visualizar esculturas de madera policromadas. Se obtendría un valor cercano a 1 en la opacidad en aquellas zonas más cercanas a los bordes entre materiales pues se ha establecido que para un gradiente 0 la opacidad sea 0, y para 2000, 1. }
	\label{fig:gradient_tf}
\end{figure}

\chapter{Diseño}

En este capítulo se mostrarán los diagramas UML \textbf{arquitectónico}, de \textbf{clases} y algunos de \textbf{secuencia} de la aplicación. Siendo los dos último generados con la aplicación \textit{Visual Paradigm for UML 13.1 Community Edition} \cite{vpp}.

\section{Arquitectura del software}

El software, escrito en C++, hace uso tanto de la CPU como de la GPU a nivel hardware y usa OpenGL como librería gráfica de bajo nivel.

Se hace uso de la librería gráfica de alto nivel VTK, de Qt como librería para crear la GUI y de Boost para usar sus algoritmos en la gestión de ficheros XML.

\begin{table}[H]
	\begin{center}
		\begin{tabular}{|l|c|c|c|c|}
			\hline
			Librerías de alto nivel  & \multicolumn{2}{c|}{Boost} & VTK        & Qt        \\ \hline
			Librerías de bajo nivel  & \multicolumn{4}{c|}{OpenGL}                         \\ \hline
			Lenguaje de programación & \multicolumn{4}{c|}{C++}                            \\ \hline
			Nivel Hardware           & \multicolumn{2}{c|}{CPU} & \multicolumn{2}{c|}{GPU} \\ \hline
		\end{tabular}
	\end{center}
	\caption{Arquitectura del software}
	\label{tab:diagrama_arquitectonico}
\end{table}

\section{Diagramas de clases}

Se presentan las distintas clases en tres diagramas distintos para ser mostrados con más claridad aunque no estén separados en paquetes como tales.

\subsection{Application}

\begin{figure}[H]
	\centering
	\includegraphics[width=10.5cm]{imagenes/diagramas/clases/Application}
	\caption{Diagrama de clases del paquete \textit{Application}}
	\label{fig:diagrama_clases_application}
\end{figure}

\subsection{Charts}

\begin{figure}[H]
	\centering
	\includegraphics[width=12.5cm]{imagenes/diagramas/clases/Charts}
	\caption{Diagrama de clases del paquete \textit{Charts}}
	\label{fig:diagrama_clases_charts}
\end{figure}

\subsection{Volume}

\begin{figure}[H]
	\centering
	\includegraphics[width=12.5cm]{imagenes/diagramas/clases/Volume}
	\caption{Diagrama de clases del paquete \textit{Volume}}
	\label{fig:diagrama_clases_volume}
\end{figure}

\section{Diagramas de secuencia}

A continuación se muestran algunos de los diagramas de secuencia de la aplicación: Aquellos que se han considerado más importantes.

\subsection{OpacityTFChart}

\begin{figure}[H]
	\centering
	\includegraphics[angle=90,width=12cm]{imagenes/diagramas/secuencia/OpacityTFChart_New}
	\caption{Diagrama de secuencia del constructor de \textit{OpacityTFChart}}
	\label{fig:diagrama_secuencia_opacitytfchart_new}
\end{figure}

\subsection{ColorTFChart}

\begin{figure}[H]
	\centering
	\includegraphics[angle=90,height=18cm]{imagenes/diagramas/secuencia/ColorTFChart_New}
	\caption{Diagrama de secuencia del constructor de \textit{ColorTFChart}}
	\label{fig:diagrama_secuencia_colortfchart_new}
\end{figure}

\subsection{ColorTransferControlPointsItem}

\begin{figure}[H]
	\centering
	\includegraphics[angle=90,width=12cm]{imagenes/diagramas/secuencia/ColorTransferControlPointsItem_MouseDoubleClickEvent}
	\caption{Diagrama de secuencia del evento de doble click de \textit{ColorTransferControlPointsItem}}
	\label{fig:diagrama_secuencia_colortransfercontrolpointsitem_mousedoubleclickevent}
\end{figure}

\subsection{Figura}

\begin{figure}[H]
	\centering
	\includegraphics[angle=90,height=18cm]{imagenes/diagramas/secuencia/Figura_New}
	\caption{Diagrama de secuencia del constructor de \textit{Figura}}
	\label{fig:diagrama_secuencia_figura_new}
\end{figure}

\begin{figure}[H]
	\centering
	\includegraphics[width=12cm]{imagenes/diagramas/secuencia/Figura_SetProperties}
	\caption{Diagrama de secuencia del método \textit{setProperties} de \textit{Figura}}
	\label{fig:diagrama_secuencia_figura_setproperties}
\end{figure}

\begin{figure}[H]
	\centering
	\includegraphics[width=12cm]{imagenes/diagramas/secuencia/Figura_ConnectComponents}
	\caption{Diagrama de secuencia del método \textit{connectComponents} de \textit{Figura}}
	\label{fig:diagrama_secuencia_figura_connectcomponents}
\end{figure}

\begin{figure}[H]
	\centering
	\includegraphics[angle=90,height=18cm]{imagenes/diagramas/secuencia/Figura_SetDicomFolder}
	\caption{Diagrama de secuencia del método \textit{setDICOMFolder} de \textit{Figura}}
	\label{fig:diagrama_secuencia_figura_setdicomfolder}
\end{figure}

\subsection{TransferFunction}

\begin{figure}[H]
	\centering
	\includegraphics[width=10cm]{imagenes/diagramas/secuencia/TransferFunction_ReadFromString}
	\caption{Diagrama de secuencia del método \textit{read} de \textit{TransferFunction}}
	\label{fig:diagrama_secuencia_transferfunction_read}
\end{figure}

\begin{figure}[H]
	\centering
	\includegraphics[width=12cm]{imagenes/diagramas/secuencia/TransferFunction_ReadData}
	\caption{Diagrama de secuencia del método \textit{readData} de \textit{TransferFunction}}
	\label{fig:diagrama_secuencia_transferfunction_readdata}
\end{figure}

\begin{figure}[H]
	\centering
	\includegraphics[width=12cm]{imagenes/diagramas/secuencia/TransferFunction_Write}
	\caption{Diagrama de secuencia del método \textit{write} de \textit{TransferFunction}}
	\label{fig:diagrama_secuencia_transferfunction_write}
\end{figure}

\subsection{MainWindow}

\begin{figure}[H]
	\centering
	\includegraphics[angle=90,height=18cm]{imagenes/diagramas/secuencia/MainWindow_New}
	\caption{Diagrama de secuencia del constructor de \textit{MainWindow}}
	\label{fig:diagrama_secuencia_mainwindow_new}
\end{figure}

\begin{figure}[H]
	\centering
	\includegraphics[angle=90,height=18cm]{imagenes/diagramas/secuencia/MainWindow_ConnectComponents}
	\caption{Diagrama de secuencia del método \textit{connectComponents} de \textit{MainWindow}}
	\label{fig:diagrama_secuencia_mainwindow_connectcomponents}
\end{figure}

\begin{figure}[H]
	\centering
	\includegraphics[angle=90,height=18cm]{imagenes/diagramas/secuencia/MainWindow_ImportDICOM}
	\caption{Diagrama de secuencia del método \textit{importDICOM} de \textit{MainWindow}}
	\label{fig:diagrama_secuencia_mainwindow_importdicom}
\end{figure}

\begin{figure}[H]
	\centering
	\includegraphics[angle=90,height=18cm]{imagenes/diagramas/secuencia/MainWindow_ExportImageFromRenderWindow}
	\caption{Diagrama de secuencia del método \textit{exportImageFromRenderWindow} de \textit{MainWindow}}
	\label{fig:diagrama_secuencia_mainwindow_exportimagefromrenderwindow}
\end{figure}

\begin{figure}[H]
	\centering
	\includegraphics[angle=90,width=12cm]{imagenes/diagramas/secuencia/MainWindow_ExportMeshToFile}
	\caption{Diagrama de secuencia del método \textit{exportMeshToFile} de \textit{MainWindow}}
	\label{fig:diagrama_secuencia_mainwindow_exportmeshtofile}
\end{figure}

\begin{figure}[H]
	\centering
	\includegraphics[angle=90,width=12cm]{imagenes/diagramas/secuencia/MainWindow_ChangeBackgroundColor}
	\caption{Diagrama de secuencia del método \textit{changeBackgroundColor} de \textit{MainWindow}}
	\label{fig:diagrama_secuencia_mainwindow_changeBackgroundColor}
\end{figure}

\begin{figure}[H]
	\centering
	\includegraphics[width=12cm]{imagenes/diagramas/secuencia/MainWindow_AddRule}
	\caption{Diagrama de secuencia del método \textit{addRule} de \textit{MainWindow}}
	\label{fig:diagrama_secuencia_mainwindow_addrule}
\end{figure}

\begin{figure}[H]
	\centering
	\includegraphics[width=12cm]{imagenes/diagramas/secuencia/MainWindow_DeleteRule}
	\caption{Diagrama de secuencia del método \textit{deleteRule} de \textit{MainWindow}}
	\label{fig:diagrama_secuencia_mainwindow_deleterule}
\end{figure}

\begin{figure}[H]
	\centering
	\includegraphics[width=12cm]{imagenes/diagramas/secuencia/MainWindow_ClearAllRules}
	\caption{Diagrama de secuencia del método \textit{clearAllRules} de \textit{MainWindow}}
	\label{fig:diagrama_secuencia_mainwindow_clearallrules}
\end{figure}

\subsection{InteractorStyleDeleter}

\begin{figure}[H]
	\centering
	\includegraphics[angle=90,width=12cm]{imagenes/diagramas/secuencia/InteractorStyleDeleter_OnLeftButtonDown}
	\caption{Diagrama de secuencia del método \textit{OnLeftButtonDown} de \textit{InteractorStyleDeleter}}
	\label{fig:diagrama_secuencia_interactorstyledeleter_onfeftbuttondown}
\end{figure}

\begin{figure}[H]
	\centering
	\includegraphics[angle=90,height=18cm]{imagenes/diagramas/secuencia/InteractorStyleDeleter_DeleteByImages}
	\caption{Diagrama de secuencia del método \textit{deleteByImages} de \textit{InteractorStyleDeleter}}
	\label{fig:diagrama_secuencia_interactorstyledeleter_deletebyimages}
\end{figure}

\begin{figure}[H]
	\centering
	\includegraphics[width=12cm]{imagenes/diagramas/secuencia/InteractorStyleDeleter_DeleteImage}
	\caption{Diagrama de secuencia del método \textit{deleteImage} de \textit{InteractorStyleDeleter}}
	\label{fig:diagrama_secuencia_interactorstyledeleter_deleteimage}
\end{figure}

\begin{figure}[H]
	\centering
	\includegraphics[width=12cm]{imagenes/diagramas/secuencia/InteractorStyleDeleter_SearchInitialVoxel}
	\caption{Diagrama de secuencia del método \textit{searchInitialVoxel} de \textit{InteractorStyleDeleter}}
	\label{fig:diagrama_secuencia_interactorstyledeleter_searchinitialvoxel}
\end{figure}

\subsection{InteractorStyleImage}

\begin{figure}[H]
	\centering
	\includegraphics[angle=90,height=18cm]{imagenes/diagramas/secuencia/InteractorStyleImage_OnMouseMove}
	\caption{Diagrama de secuencia del método \textit{OnMouseMove} de \textit{InteractorStyleImage}}
	\label{fig:diagrama_secuencia_interactorstyleimage_onmousemove}
\end{figure}
\chapter{Implementación}

En este capítulo se hablará del desarrollo del software describiendo la plataforma de desarrollo y despliegue, así como las distintas decisiones tomadas a la hora de implementar y las fases de desarrollo.

\section{Plataforma de desarrollo}

En un principio, la idea fue desarrollar usando una distribución Linux un \textbf{software multiplataforma} ayudándose, para ello, de CMake \cite{cmake} que a partir de los archivos fuentes puede crear \textit{makefiles} para distintas plataformas.

No obstante, unos problemas detectados con los \textit{drivers} de mi GPU provocaban que no se pudiese integrar Qt \cite{qt} con VTK \cite{vtk} ya que el \textit{widget} especializado para esta tarea (\texttt{QVTKWidget}) no funcionaba. Tras unos días intentando solucionar los problemas sin éxito se decidió migrar a Windows donde no surgió ningún problema parecido al que se dio en Linux.

Más tarde además de Qt y VTK se decidió utilizar la librería Boost \cite{boost} para facilitar el tratamiento de ficheros XML.

Se ha utilizado \textbf{Windows 10 Pro} (64 bits) con el siguiente software y librerías:

\begin{itemize}
	\item CMake 3.4.1
	\item Visual Studio Community 2013
	\item Qt5.5.1
	\item VTK 7.0.0
	\item Boost 1.60.0
\end{itemize}

Cuando se empezó a desarrollar el software la versión más nueva de VTK era la \textbf{6.3.0} y se empezó utilizando ésta. No obstante, tenía un pequeño fallo que afectaba mucho a la aplicación y es que si se hacía uso de la GPU en el \textit{ray casting} al renderizar el volumen, \textbf{la opacidad gradiente no se computaba}. Por suerte, a principios del mes de Febrero, se lanzó una nueva versión de VTK, la \textbf{7.0.0}, que solucionaba este fallo.

Pero los problemas no acababan aquí, y es que, con la nueva versión, el programa no llegaba a funcionar y \textbf{se bloqueaba nada más iniciarse}. Detectar el fallo era complicado y al no ser todavía demasiado complejo era más rápido detectar de dónde venía el error si se volvía a crear desde cero agregando uno a uno cada componente.

\textbf{El fallo lo estaba dando el plano de corte} y es que en la versión anterior al activarlo, si no tenía ningún volumen con el que cortar, no producía ningún fallo, pero con la nueva versión era un requisito imprescindible. La solución era bastante sencilla: habilitar el plano cuando se cargase el volumen en lugar de tenerlo habilitado desde su propia construcción.

\section{Instalación y configuración}

\subsection{Entorno de desarrollo}

\subsubsection{Visual Studio Community 2013}

\begin{itemize}
	\item Descargar Visual Studio Community 2013 desde su \href{https://www.visualstudio.com/es-es/downloads/download-visual-studio-vs.aspx}{web oficial} e instalar.
\end{itemize}

\subsubsection{Qt5.5.1}

\begin{itemize}
	\item Descargar Qt5.5.1 desde \href{http://download.qt.io/official_releases/qt/5.5/5.5.1/qt-opensource-windows-x86-msvc2013-5.5.1.exe}{este enlace} de su web e instalar.
	\item Crear una nueva variable de entorno con nombre \texttt{QTDIR} y valor \texttt{C:\textbackslash\\ Qt\textbackslash Qt5.5.1} (directorio raíz de la versión instalada).
	\item Agregar al PATH la siguiente dirección \texttt{C:\textbackslash Qt\textbackslash Qt5.5.1\textbackslash 5.5\textbackslash msvc2013\\ \textbackslash bin}.
\end{itemize}

\subsubsection{CMake 3.4.1}

\begin{itemize}
	\item Descargar CMake 3.4.1 desde \href{https://cmake.org/files/v3.4/cmake-3.4.1-win32-x86.exe}{este enlace} de su web e instalar (al instalar se recomienda marcar la opción de agregar al PATH de todos los usuarios para no tener que hacerlo manualmente).
\end{itemize}

\subsection{Compilar librerías}

\subsubsection{VTK 7.0.0}

\begin{itemize}
	\item Descargar VTK 7.0.0 desde \href{http://www.vtk.org/files/release/7.0/VTK-7.0.0.zip}{este enlace} de su web oficial y extraer en \texttt{C:\textbackslash VTK\textbackslash 7.0.0\textbackslash src}.
	\item Abrir CMake y completar:
	\begin{itemize}
		\item src: \texttt{C:\textbackslash VTK\textbackslash 7.0.0\textbackslash src}
		\item build: \texttt{C:\textbackslash VTK\textbackslash 7.0.0\textbackslash build\textbackslash vs12}
	\end{itemize}
	\item Elegir como generador \textit{Visual Studio 12 2013}.
	\item Presionar configurar.
	\item Una vez generado seleccionar los siguientes campos:
	\begin{itemize}
		\item \texttt{BUILD\_SHARED\_LIBS}
		\item \texttt{Module\_vtkGUISupportQt}
		\item \texttt{Module\_vtkGUISupportQtOpenGL}
		\item \texttt{Module\_vtkGUISupportQtSQL}
		\item \texttt{Module\_vtkGUISupportQtWebkit}
		\item \texttt{Module\_vtkRenderingQt}
		\item \texttt{Module\_vtkViewsQt}
		\item \texttt{Module\_vtkDICOM}
		\item \texttt{VTK\_Group\_Qt}
	\end{itemize}
	\item Agregar dos entradas:
	\begin{itemize}
		\item \texttt{QT\_QMAKE\_EXECUTABLE:PATHFILE=\\C:\textbackslash Qt\textbackslash Qt5.5.1\textbackslash 5.5\textbackslash msvc2013\textbackslash bin\textbackslash qmake.exe}
		\item \texttt{CMAKE\_PREFIX\_PATH:PATH=C:\textbackslash Qt\textbackslash Qt5.5.1\textbackslash 5.5\textbackslash msvc2013\textbackslash}
	\end{itemize}
	\item Presionar en configurar y aparecerá un error, habrá que elegir como versión de Qt la 5. Elegirla y volver a configurar.
	\item Configurar hasta que no aparezca ningún campo en rojo.
	\item Una vez configurado todo, pulsar en generar. Esto creará una serie de archivos en \texttt{C:\textbackslash VTK\textbackslash 7.0.0\textbackslash build\textbackslash vs12}.
	\item Abrir \texttt{VTK.sln}.
	\item Construir en modo \textit{Release} y esperar unos minutos a que termine.
	\item Copiar los archivos \texttt{QVTKWidgetPlugin.lib} y \texttt{QVTKWidgetPlugin.dll} que se encuentran en \texttt{C:\textbackslash VTK\textbackslash 7.0.0\textbackslash build\textbackslash vs12\textbackslash lib\textbackslash Release} y \texttt{C:\textbackslash\\  VTK\textbackslash 7.0.0\textbackslash build\textbackslash vs12\textbackslash bin\textbackslash Release} respectivamente en \texttt{C:\textbackslash Qt\textbackslash\\ Qt5.5.1\textbackslash 5.5\textbackslash msvc2013\textbackslash plugins\textbackslash designer} (Si no se encuentran los archivos, comprobar que en CMake se marcó la opción \texttt{BUILD\_SHARED\_\\ LIBS}). Esto hará que desde Qt Designer se pueda crea un \texttt{QVTKWidget}.
	\item Construir en modo \textit{Debug}.
	\item Crear una nueva variable de entorno con nombre: \texttt{VTK\_DIR} y valor: \texttt{C:\textbackslash VTK\textbackslash 7.0.0\textbackslash build\textbackslash vs12}.
	\item Agregar al Path la siguiente dirección: \texttt{C:\textbackslash VTK\textbackslash 7.0.0\textbackslash build\textbackslash vs12\textbackslash bin\\ \textbackslash Release}.
\end{itemize}

\subsubsection{Boost 1.60.0}

\begin{itemize}
	\item Descargar Boost 1.60.0 desde \href{http://sourceforge.net/projects/boost/files/boost/1.60.0/}{este enlace} de su web oficial.
	\item Descomprimir en cualquier lugar, abrir la consola de comandos de Visual Studio y moverse al lugar donde ha sido extraído.
	\item Escribir \texttt{boostrap.bat} para generar el \texttt{Boost.Build}.
	\item Compilar con: \texttt{b2 toolset=msvc-12.0 --build-type=complete \\ --abbreviate-paths architecture=x86 address-model=64 \\ install -j4}.
	\item Agregar al proyecto de Visual Studio: 
\end{itemize}

\subsection{Configurar proyecto}

Una vez generado el proyecto realizar los siguientes cambios en la configuración: 
\begin{itemize}
	\item En \textit{Project Properties} ir a \textit{Configuration Properties $ \rangle $ C/C++ $ \rangle $ General $ \rangle $ Additional Include Directories} y añadir el directorio \texttt{C:\textbackslash Boost\textbackslash \\  include\textbackslash boost-1\_60}.
	\item En \textit{Project Properties} ir a \textit{Configuration Properties $ \rangle $ Linker $ \rangle $ Additional Library Directories} y añadir el directorio \texttt{C:\textbackslash Boost\textbackslash lib}.
	\item En \textit{Project Properties} ir a \textit{Configuration Properties $ \rangle $ Linker $ \rangle $ System} y:
	\begin{itemize}
		\item En \textit{Subsystem} seleccionar la opción: \\ \texttt{Windows (/SUBSYSTEM:WINDOWS)}.
		\item En \textit{Enable Large Adresses} seleccionar la opción: \\ \texttt{Yes (/LARGEADRESSESAWARE)}.
	\end{itemize}
\end{itemize}

\section{Conceptos clave en Volume Rendering}

\subsection{Técnicas de renderizado}

A la hora de renderizar un conjunto de datos volumétricos para obtener una imagen en 3D, se pueden utilizar distintas técnicas y VTK proporciona una serie de clases para su uso:

\begin{itemize}
	\item \textbf{\textit{Marching cubes}}: Con este algoritmo se obtiene una malla poligonal de una isosuperficie a partir de un conjunto de datos volumétrico (Figura \ref{fig:marching_cubes_head}) \cite{marching_cubes}. Se puede usar en VTK con \texttt{vtkMarchingCubes}.
	\begin{figure}[H]
		\centering
		\includegraphics[width=6cm]{imagenes/marching_cubes_head}
		\caption{Cabeza extraída de 150 cortes obtenidos por una IRM usando \textit{marching cubes} (sobre 150.000 triángulos). Imagen extraída de \url{https://en.wikipedia.org/wiki/File:Marchingcubes-head.png}}
		\label{fig:marching_cubes_head}
	\end{figure}
	
	\item \textbf{Texturas 2D}: Se utilizan planos de corte alineados a los ejes de coordenadas. Por lo que se tendría una serie de cortes sobre el plano sagital, otra sobre el coronal y otra sobre el axial. Se realiza una interpolación bilineal para obtener la imagen final (Figura \ref{fig:texturas2d} \cite{intro_medical_vtk_bioimage}). Se puede usar en VTK con \texttt{vtkVolumeTextureMapper}.
	\begin{figure}[H]
		\centering
		\includegraphics[width=10cm]{imagenes/texturas2d}
		\caption{Esquema del proceso de renderizado usando texturas 2D. Imagen extraída del apéndice B del libro \textit{An Introduction to Programming for Medical Image Analysis with the Visualization Toolkit} \cite{intro_medical_vtk_bioimage}}
		\label{fig:texturas2d}
	\end{figure}
	
	\item \textbf{Texturas 3D}: Esta técnica es similar a la anterior, pero ahora los datos se cargan en una textura 3D y los cortes se dibujan paralelos a la dirección de vista. A diferencia de las texturas 2D, usa interpolación trilineal y no es necesario tener almacenado en memoria tres copias de los mismos datos (Figura \ref{fig:texturas3d}) \cite{intro_medical_vtk_bioimage}. Se puede usar en VTK con \texttt{vtkVolumeTextureMapper3D}.
	\begin{figure}[H]
		\centering
		\includegraphics[width=10cm]{imagenes/texturas3d}
		\caption{Esquema del proceso de renderizado usando texturas 3D. Imagen extraída del apéndice B del libro \textit{An Introduction to Programming for Medical Image Analysis with the Visualization Toolkit} \cite{intro_medical_vtk_bioimage}}
		\label{fig:texturas3d}
	\end{figure}
	
	\item \textbf{\textit{Ray casting}}: Es una técnica en la que para cada pixel de la imagen se lanza un rayo que atraviesa el volumen. Para cada voxel se obtiene su color y opacidad usando una función de transferencia. Cuando el rayo sale del volumen se calcula el color y opacidad del pixel como el acumulado por el rayo. Existe una versión de este algoritmo que hace uso de la GPU para acelerar ostensiblemente el tiempo de la operación (Figura \ref{fig:volume_ray_casting}) \cite{intro_medical_vtk_bioimage}. Se puede usar en VTK con \texttt{vtkFixedVolumeRayCastMapper}, \texttt{vtkVolumeRayCastMapper} (usan CPU), \texttt{vtkGPUVolumeRayCastMapper} (usa GPU) y \texttt{vtkSmartVolumeMapper} (según el contexto usa CPU o GPU).
	\begin{figure}[H]
		\centering
		\includegraphics[width=12.5cm]{imagenes/volume_ray_casting}
		\caption{Esquema del proceso de \textit{ray casting}. Imagen extraída de \url{https://en.wikipedia.org/wiki/File:Volume_ray_casting.png}}
		\label{fig:volume_ray_casting}
	\end{figure}
\end{itemize}

De entre todas estas técnicas, se podrían descartar rápidamente la de \textit{marching cubes}: pues tan solo trabaja con isosuperficies, y la de texturas 2D: pues la opción de texturas 3D es más rápida y usa menos recursos. Sin embargo, la opción de \textit{marching cubes} será útil para poder crear una malla de triángulos que se pueda exportar a un formato con el que luego pueda ser imprimida en 3D.

Por tanto ya solo habría que elegir entre texturas 3D o \textit{ray casting}. Hasta hace unos años, VTK no proporcionaba un algoritmo de \textit{ray casting} que usase la GPU. Por tanto la opción habría sido sencilla, pero durante los últimos años han trabajado en esto haciendo del \textit{ray casting} la opción preferible.

\subsection{Volume Mapper}

Para poder visualizar un volumen con VTK mediante Direct Volume Rendering (DVR), necesitamos un \textit{Volume Mapper}. La librería nos ofrece varias alternativas:

\begin{itemize}
	\item \texttt{vtkAMRVolumeMapper}
	\item \texttt{vtkFixedVolumeRayCastMapper}
	\item \texttt{vtkGPUVolumeRayCastMapper}
	\item \texttt{vtkSmartVolumeMapper}
	\item \texttt{vtkVolumeRayCastMapper}
	\item \texttt{vtkVolumeTextureMapper}
	\item \texttt{vtkVolumeTextureMapper3D}
\end{itemize}

Entre esta lista tenemos algunos que utilizan o texturas o \textit{ray casting}, o la CPU o la GPU. Pero hay uno que es especial con respecto al resto. Se trata de \texttt{vtkSmartVolumeMapper}.

Este \textit{Volume Mapper} es una versión mejorada del \texttt{vtkGPUVolumeRayCast Mapper} por lo que utiliza la GPU (si el dispositivo cuenta con una) y la técnica de \textit{ray casting}. Además cuenta con nuevas características con respecto al resto, como el poder definir infinitos planos de corte para poder ver el interior del volumen \cite{smart_volume_mapper}.

Por tanto, el \textit{Volume Mapper} utilizado será el \texttt{vtkSmartVolumeMapper}.

\subsection{Función de transferencia}

La función de transferencia es la encargada de dar a un valor de intensidad las propiedades de color y opacidad que le corresponden para la visualización del volumen.

En VTK la función de transferencia forma parte de la clase \texttt{vtkVolume Property} \cite{vtk_example_medical4}. Para ello proporciona otras dos clases:

\begin{itemize}
	\item \texttt{vtkColorTransferFunction}: Para definir el color. Se enlaza a \texttt{vtk VolumeProperty} con el método \texttt{SetColor}.
	\item \texttt{vtkPiecewiseFunction}: Para definir la opacidad (tanto escalar como gradiente). La opacidad escalar se enlaza a \texttt{vtkVolumeProperty} con el método \texttt{SetScalarOpacity} y la gradiente con \texttt{SetGradientOpacity}.
\end{itemize}

Podemos, por tanto, diferenciar tres partes fundamentales en la función de transferencia, la encargada de dar la propiedad de color y las dos de dar la propiedad de opacidad. Ambas trabajan de forma independiente. Es decir, cuando se define un punto en una de ellas, no tiene por qué definirse en la otra.

\subsubsection{Color}

Para definir esta función (\texttt{vtkColorTransferFunction}), hay que agregar puntos para valores de intensidad a los que se les asignará un color. VTK se encargará de interpolar entre un punto y otro (Figura \ref{fig:color_tf}). 

Por defecto, cuando no hay ningún punto, a todos los valores de intensidad les corresponderá un color negro. De forma parecida se comporta cuando solo hay un punto pero en lugar de negro, les corresponderá el color del punto que se ha definido.

VTK permite trabajar tanto con HSV como con RGB y para añadir un punto hay que utilizar \texttt{AddHSVPoint} o \texttt{AddRGBPoint}. A estos métodos se les pasa un primer parámetro en coma flotante con el valor de intensidad donde se establecerá ese punto y otros tres con las distintas exponentes (\textit{hue}, \textit{saturation}, \textit{brightness} o \textit{red}, \textit{green}, \textit{blue}).

\begin{figure}[H]
	\centering
	\includegraphics[width=12.5cm]{imagenes/color_tf}
	\caption{Parte de color de la función de transferencia del \textit{preset} \textit{CT-WoodSculpture} creado para visualizar esculturas de madera policromadas. Dos puntos definen el color. Uno en -750 con un tono más oscuro y otro en -350 con un tono más claro. El estuco se pinta con un color gris claro y también viene definido por dos puntos: -200 y 2750. Finalmente, el metal se verá con un tono gris oscuro definido con un punto en 3000.}
	\label{fig:color_tf}
\end{figure}

\subsubsection{Opacidad}

El valor de opacidad se obtendría como el \textbf{producto de la opacidad escalar por la gradiente}. Si no se define alguna de las dos, se definiría como un valor constante de 1 para que solo se viese el resultado de la que sí está definida.

\paragraph{\textbf{Opacidad escalar:}} Con el color no bastaría, pues si comprobásemos ahora añadiéndole tan solo el \texttt{vtkColorTransferFunction} al \texttt{vtkVolumeProperty} observaríamos que no se pinta nada en pantalla. Esto es porque por defecto, al no tener ningún punto la función de opacidad (\texttt{vtkPiecewiseFunction}) es una constante con valor 0 (transparente). 

Para definir esta función se trabaja de forma parecida a como se hace con el color, añadiendo puntos. El método que hay que utilizar es \texttt{AddPoint} al que se le pasan dos parámetros en coma flotante. El primero con el valor de intensidad y el segundo con la opacidad en ese punto. Para obtener los valores en puntos intermedios, se interpola entre los dos puntos en los que está. De forma que si para el valor de intensidad 100 hemos definido una opacidad de 0.5 y para el de 200 1, al valor de intensidad 150 le corresponderá 0.75.

Combinando color y opacidad escalar podemos obtener una función de transferencia para visualizar nuestro volumen (Figura \ref{fig:opacity_tf}), pero para obtener mejores resultados, habrá que utilizar la opacidad gradiente. 

\begin{figure}[H]
	\centering
	\includegraphics[width=12.5cm]{imagenes/opacity_tf}
	\caption{Parte de opacidad escalar de la función de transferencia del \textit{preset} \textit{CT-WoodSculpture} creado para visualizar esculturas de madera policromadas. Se pueden observar tres regiones. La primera corresponde a la madera, la segunda al estuco y la última al metal}
	\label{fig:opacity_tf}
\end{figure}

\paragraph{\textbf{Opacidad gradiente:}} La opacidad gradiente utiliza el vector gradiente para dar el valor de opacidad. Con éste se puede conseguir \textbf{dar un mayor valor a regiones de los bordes, así como menor a regiones planas} es decir, donde no varía el valor de intensidad de sus vecinos de alrededor.

El gradiente se mide como la cantidad que varía la intensidad en una unidad de distancia. Este cálculo del gradiente lo realiza VTK cuando genera el volumen de forma transparente sin que haya que añadir nada al código.

Para poder definir la función de la opacidad gradiente, al igual que con las demás, hay que añadir puntos con la misma función que se usaba con la opacidad escalar (\texttt{AddPoint}). 

\begin{figure}[H]
	\centering
	\includegraphics[width=12.5cm]{imagenes/gradient_tf}
	\caption{Parte de opacidad gradiente de la función de transferencia del \textit{preset} \textit{CT-WoodSculpture} creado para visualizar esculturas de madera policromadas. Se obtendría un valor cercano a 1 en la opacidad en aquellas zonas más cercanas a los bordes entre materiales pues se ha establecido que para un gradiente 0 la opacidad sea 0, y para 2000, 1. }
	\label{fig:gradient_tf}
\end{figure}

\subsection{Escala Hounsfield}

Como ya se ha explicado, para desarrollar la función de transferencia con la que se visualiza el volumen, juega un papel muy importante el valor de densidad del material. 

Este valor se encuentra en unas unidades conocidas como Unidades Hounsfield (HU) en honor al ingeniero Godfrey Newbold Hounsfield, inventor del primer escáner TAC con el que ganó el Premio Nobel de Fisiología o Medicina en 1979.

La Escala Hounsfield no es más que la transformación de la escala de coeficientes de atenuación lineal de rayos X a una nueva en relación al valor del agua destilada en condiciones normales de presión y temperatura.

El valor de HU de un material viene dado por la siguiente fórmula:

\[ HU = 1000 \times \frac{\mu_{mat}-\mu_{agua}}{\mu_{agua}} \]

Donde $ \mu_{mat} $ es el coeficiente de atenuación lineal del material y $ \mu_{agua} $ el del agua.

Por tanto, el valor teórico del agua será 0 HU.

El rango de valores de la escala va desde -1024 HU hasta 3071 HU. 4096 valores representados mediante 12 bits.

\begin{table}[H]
	\begin{center}
		\begin{tabular} {l|c}
			\hline
			Material & HU \\ \noalign{\hrule height 1pt}
			Aire & -1000 \\ \hline
			Madera & -750 a -350 \\ \hline
			Estuco & 200 a 1000 \\ \hline
			Metal & 2900 a 3000 \\ \hline
		\end{tabular}
		\caption{Valores en HU de distintos materiales presentes en imágenes de esculturas de madera}
		\label{tab:materials_hu}
	\end{center}
\end{table}

\section{Fases de desarrollo}

Al seguirse un desarrollo evolutivo basado en un prototipo funcional, se ha ido creando poco a poco y añadiendo componentes conforme se iban completando y probando lo que ya se había desarrollado:

\subsection{Lectura de imágenes DICOM}

El primer paso fue poder \textbf{leer una imagen DICOM}. Para ello, se utilizó el ejemplo \href{http://www.vtk.org/Wiki/VTK/Examples/Cxx/IO/ReadDICOM}{ReadDICOM} de la web de ejemplos de VTK que hace uso de \texttt{vtkDICOMImageReader} para leer la imagen.

A continuación, se pasó a \textbf{leer una serie de imágenes}. En la lista de ejemplos de VTK también había uno que realizaba esta operación: \href{http://www.vtk.org/Wiki/VTK/Examples/Cxx/IO/ReadDICOMSeries}{ReadDICOMSeries}, pero era hora de integrarlo con Qt. Y aquí es donde surgió el problema comentado anteriormente con los \textit{drivers} de la GPU.

Tras migrar a Windows y lograr crear un \textbf{pequeño programa en Qt} (Figura \ref{fig:read_dicom_series_qt}) con el que visualizar una serie de imágenes (desplazándose entre ellas con un slider) pasé a la siguiente fase, una de las más importantes, la de la reconstrucción volumétrica.

\begin{figure}[H]
	\centering
	\includegraphics[width=10cm]{imagenes/read_dicom_series_qt}
	\caption{Programa sencillo para visualizar una serie de imágenes DICOM}
	\label{fig:read_dicom_series_qt}
\end{figure}

\subsection{Reconstrucción volumétrica}

El primer paso de esta fase fue elegir cómo \textbf{renderizar el volumen}. Como ya se ha explicado detalladamente con anterioridad, se eligió el \textit{ray casting} que usa GPU haciendo uso de \texttt{vtkSmartVolumeMapper}.

Una parte fundamental del \textit{ray casting} es la \textbf{función de transferencia} con la que mapea valores. Para poder crearla se ha hecho uso del software 3DSlicer \cite{slicer} para poder ver los valores de densidad de los distintos materiales de la escultura. Con esto se creó una función de transferencia bastante básica con la que poder hacer el primer renderizado (Figura  \ref{fig:primer_renderizado}).

\begin{figure}[H]
	\centering
	\includegraphics[width=5cm]{imagenes/primer_renderizado}
	\caption{Primer renderizado sobre la figura de San Juan Evangelista con una función de transferencia que no utilizaba la opacidad gradiente}
	\label{fig:primer_renderizado}
\end{figure}

Una vez comprobado que se estaba realizando bien el renderizado, se debía mejorar la función de transferencia. Para ello se creo una \textbf{barra de herramientas} (Figura \ref{fig:gui_inicial_tf}) con la que poder cambiar esta función de transferencia añadiendo puntos a ésta. Pese a no ser muy amigable para el usuario, era un primer prototipo que ayudaría a crear una función de transferencia con la que poder seguir trabajando.

\begin{figure}[H]
	\centering
	\includegraphics[width=7cm]{imagenes/gui_inicial_tf}
	\caption{Barra de herramientas con la que añadir y quitar puntos a la función de transferencia}
	\label{fig:gui_inicial_tf}
\end{figure}

\subsection{Generación de cortes}

Antes de mejorar la forma en la que se edita la función de transferencia se pasó a realizar una de las partes más importantes. La de la poder ver cortes de la figura. Para empezar se creó un nuevo \textit{widget} con el que se combinó lo implementado anteriormente para visualizar cortes (Figura \ref{fig:read_dicom_series_qt}). 

Pero de esta forma no se estaban generando cortes sino visualizando las propias imágenes con las que se reconstruía el volumen. La idea era poder \textbf{crear un corte en la figura dado por un plano arbitrario}. Para ello se hizo uso de \texttt{vtkImagePlaneWidget} modificando esta clase para que se renderizase el corte al mismo tiempo que se movía el plano. Este plano se conectaba con los datos del volumen (\texttt{vtkImageData}) para mostrar la salida en un \texttt{vtkImageViewer2} (Figura \ref{fig:primer_plano_de_corte}).

\begin{figure}[H]
	\centering
	\includegraphics[width=12cm]{imagenes/primer_plano_de_corte}
	\caption{Primera implementación del plano que corta la figura y renderiza el corte en otro \textit{widget}}
	\label{fig:primer_plano_de_corte}
\end{figure}

Resultó sencillo añadir a continuación mejoras a este plano, como colocarlo en \textbf{posiciones por defecto} (axial, coronal y sagital) y visualizar la salida con el \textbf{mismo color que la función de transferencia}.

Además se reescribieron métodos de eventos de ratón para eliminar los innecesarios que ya traían las clases de VTK.

\subsection{Guardar imágenes}

Otra funcionalidad básica era la de poder \textbf{guardar imágenes de ambos visores}. VTK ofrece con las distintas subclases de \texttt{vtkImageWriter} una serie de clases con las que exportar imágenes en distintos formatos. Se han utilizado tanto \texttt{vtkJPEGWriter} como \texttt{vtkPNGWriter} para darle al usuario la posibilidad de guardar la imagen tanto en JPG como en PNG. 

\subsection{Editar función de transferencia}

Con el software cada vez más completo, era el momento de mejorar la \textbf{edición de función de transferencia}. La implementación anterior (Figura \ref{fig:gui_inicial_tf}) no ofrecía retroalimentación visual al usuario de cómo era la curva de la función o la paleta de colores. Además, añadir y editar puntos era un trabajo tedioso.

Resultaba imprescindible cambiarlo y la forma más cómoda de editar la función era trabajar sobre la misma. Es decir, \textbf{mostrar la función en una gráfica} con cada uno de sus puntos y poder mover, añadir o borrarlos interactuando directamente con la gráfica.

Explorando entre las clases que proporcionaba VTK se encontraron:

\begin{itemize}
	\item \texttt{vtkColorTransferFunctionItem}: Muestra la función de transferencia de color como un \texttt{vtkPlot} que se puede añadir a una \texttt{vtkChartXY}.
	\item \texttt{vtkPiecewiseFunctionItem}: Muestra la función de transferencia de opacidad como un \texttt{vtkPlot} que se puede añadir a una \texttt{vtkChartXY}.
	\item \texttt{vtkColorTransferControlPointsItem}: Muestra y permite modificar los puntos de la función de transferencia de color como un \texttt{vtkPlot} que se puede añadir a una \texttt{vtkChartXY}.
	\item \texttt{vtkPiecewiseControlPointsItem}: Muestra y permite modificar los puntos de la función de transferencia de opacidad como un \texttt{vtkPlot} que se puede añadir a una \texttt{vtkChartXY}.
\end{itemize}

No obstante, había que reescribir algunos métodos de estas clases para poder adaptarlas al software. Había que eliminar algunos eventos innecesarios y agregar nuevas acciones a otros como volver a renderizar el volumen cada vez que se cambie algún punto.

Llevar a cabo este trabajo fue bastante costoso pero finalmente se logró proporcionar al usuario una interfaz gráfica para editar la función de transferencia intuitiva y fácil de utilizar (Figura \ref{fig:gui_final_tf}).

\begin{figure}[H]
	\centering
	\includegraphics[width=10cm]{imagenes/gui_final_tf}
	\caption{Interfaz para editar la función de transferencia}
	\label{fig:gui_final_tf}
\end{figure}

\subsection{Importar y exportar función de transferencia}

Con la nueva interfaz iba a resultar sencillo crear una batería de funciones de transferencia, por tanto, llegó el momento de almacenarlas de alguna forma para luego poder usarlas.

Para esto se decidió utilizar XML almacenándolas con un formato en el que el software pudiese exportar e importarlas (Código \ref{code:xmlschema}).

\lstinputlisting[caption={Descripción de formato XML utilizado usando XML Schema}, label=code:xmlschema, style=XML]{misc/tf-xmlschema.xml}

Por ejemplo, la función de transferencia principal utilizada en este formato XML sería la siguiente (Código \ref{code:xmltf})

\lstinputlisting[caption={Ejemplo de función de transferencia en el formato XML descrito}, label=code:xmltf, style=XML]{misc/ct-woodsculpture.xml}

Para poder \textbf{gestionar estos ficheros}, se decidió utilizar una librería. Había muchas opciones, pero lo que se quería gestionar era bastante básico por lo que utilizar una demasiado compleja podría resultar contraproducente ya que se tardaría bastante tiempo en aprender a utilizarla.

Finalmente se optó por \textbf{Boost}, una librería muy extensa que cuenta con un pequeño gestor de ficheros XML. 

A partir de un fichero XML, Boost transforma la información a un \texttt{struct} en forma de árbol de forma que luego se puede recorrer fácilmente obteniendo la información deseada.

\subsection{Realizar medida}

Una de las primeras ideas que surgieron para añadir al software era la de poder realizar medidas. Pues esto podría resultar de mucha utilidad para los restauradores.

En poco tiempo se consiguió, gracias a la clase \texttt{vtkDistanceWidget} añadir una regla con la que \textbf{realizar medidas} (Figura \ref{fig:primera_regla}).

\begin{figure}[H]
	\centering
	\includegraphics[width=8cm]{imagenes/primera_regla}
	\caption{Primera implementación de regla para medir. Solo se podía utilizar una en el visor de cortes}
	\label{fig:primera_regla}
\end{figure}

Lo ideal sería poder añadir más de una regla, pero antes de llevar esta mejora a cabo, se decidió pasar a otra quizás más importante.

\subsection{Borrar partes innecesarias}

Al hacer \textit{ray casting} se mapean los datos de la imagen en valores de color y opacidad gracias a la función de transferencia. Esto hace que se puedan diferenciar distintos materiales. Pero hay materiales con compuestos similares que pueden aparecer pues se encontraban en el escáner a la hora de realizarse la tomografía.

Hablo de \textbf{la camilla donde está apoyada la figura} que tiene zonas en las que hay un material con una densidad similar al estuco y otra con uno similar a la madera. Por lo que, al visualizarse, aparecen. Aunque no tienen nada que ver con la figura. Y no solo sobran, también molestan pues en el caso de una capa de la camilla, impide ver la figura a las espaldas (Figura \ref{fig:necesario_borrar}).

\begin{figure}[H]
	\centering
	\includegraphics[width=8cm]{imagenes/necesario_borrar}
	\caption{La camilla tapa la espalda de la figura e impide verla}
	\label{fig:necesario_borrar}
\end{figure}

Es por tanto necesario proveer al software de una herramienta para poder \textbf{borrar estas partes}.

En un principio se pensó hacer un borrado en el que a partir de un punto seleccionado por el usuario, se \textbf{extendiese en 3D} comprobando si alguno de los puntos de alrededor tiene un valor similar para seguir extendiéndose recursivamente. Pero la camilla pegada a la figura tiene zonas donde las zonas de la camilla y la figura tienen un valor muy similar y esto provocaría que borrase en el interior de la figura.

La segunda capa de camilla, la más gruesa con valores de estuco (puede verse más blanca en las imágenes), está separada de la figura por un material con valores cercanos a los del aire y que, por tanto, no se muestran.

Esta situación hizo que se cambiase un poco el planteamiento y se borrase no por zonas con valor parecido sino por \textit{islas}. Entendiendo isla como una zona separada de las demás.

Se empezó a implementar el \textbf{algoritmo recursivo} descrito anteriormente, pero usando una pila de puntos a analizar en un bucle que continuase hasta que estuviese vacía en lugar de usar recursividad como tal pues podría agotar la pila de llamadas a funciones.

Sin embargo esto no solucionaba los problemas con el hardware porque, aunque no se agotase la pila de llamadas, la pila de puntos que se utilizaba se hacía demasiado grande. Tanto que agotaba la memoria utilizada por el programa.

Era necesario hacer un cambio y se pasó de extenderse en 3D a extenderse en 2D imagen por imagen. Pasando de introducir 26 puntos en cada iteración a 8 con un tamaño total de $ res_{x} \times res_{y} $ que hace que las coordenadas de una imagen de $ 1024px \times 1024px $ tenga espacio de sobra en la memoria de la pila sin desbordarse.

Este cambio no solo era más óptimo en cuanto a utilización de memoria, sino que resultaba bastante más rápido. Y, aunque en ocasiones, no borre la isla con un solo click, el proceso de borrado tarda escasos segundos (Código \ref{code:pseudo_delete}).

\begin{lstlisting}[style=C, label=code:pseudo_delete, caption={Pseudocódigo del borrado}]
void deletByImages(data, point, bounds) {
	deleteImage(data, point, bounds)
	z = point.z + 1
	while (z < bounds.z.max) {
		point.z = z
		deleteImage(data, point, bounds)
		z++
	}
	z = point.z - 1
	while (z >= bounds.z.min) {
		point.z = z
		deleteImage(data, point, bounds)
		z--
	}
}

void deleteImage(data, point, bounds) {
	z = point.z
	xy = {point.x, point.y}
	if (data(point) < AIR_HU) {
		point = searchInitialVoxel(data, point, bounds)
	}
	stack.push(xy)
	while (!stack.empty()) {
		xy = stack.pop()
		if (isInside(xy, bounds)) {
			point = {xy.x, xy.y, z}
			if (data(point) >= MIN_AIR) {
				data(point) = AIR_HU
				stack.push(pointsAround(point))
			}
		}
	}
}
\end{lstlisting}

Gracias a este algoritmo de borrado se puede eliminar fácil y rápidamente la camilla que no está pegada a la figura y, al ser la capa que está pegada muy fina y tener activada la opacidad gradiente, se puede ver las espaldas de la figura a diferencia de antes (Figura \ref{fig:camilla_borrada}).

\begin{figure}[H]
	\centering
	\includegraphics[width=8cm]{imagenes/camilla_borrada}
	\caption{La camilla ha sido eliminada y permite ver la espalda a diferencia de antes (Figura \ref{fig:necesario_borrar})}
	\label{fig:camilla_borrada}
\end{figure}

No obstante, y aunque esta capa que se sigue viendo no molesta tanto, a la hora de realizar el escáner podría resultar útil colocar un material que se sepa que tiene un valor de densidad similar al aire entre la figura y la camilla para que así pueda ser borrada por completa la camilla.

\subsection{Exportar malla de triángulos}

Una de las últimas ideas de funcionalidad a introducir en el software es la de la \textbf{generación de una malla de triángulos} del modelo que pudiese ser exportada en formato STL.

Para ello, como se avanzó con anterioridad, se utiliza la técnica de \textit{marching cubes} y la clase \texttt{vtkMarchingCubes} que VTK proporciona para ello.

Por lo que se introduce otro \textit{widget} en la aplicación para visualizar la malla generada a partir de un valor de isosuperficie dado por el usuario.

De esta forma se puede obtener, por ejemplo, un modelo de los clavos de una figura \ref{fig:malla_clavos} que luego podrían imprimirse con una impresora 3D.

\begin{figure}[H]
	\centering
	\includegraphics[width=8cm]{imagenes/malla_clavos}
	\caption{Malla generada con un valor de isosuperficie de 2976 HU para poder extraer los clavos}
	\label{fig:malla_clavos}
\end{figure}

\subsection{Mostrar valor de densidad}

Al empezar a crear la función de transferencia, ya se comentó la utilización de otro software para comprobar \textbf{el valor escalar de cada pixel} de la imagen para poder ver entre qué valores se movía cada material. Es por ello que pareció interesante añadir en la interfaz esta información.

Por lo que, como última idea de mejora, y aprovechando lo aprendido de los \textit{pickers} durante la implementación del borrado, se añadió a la interfaz una etiqueta donde se muestra el valor del pixel sobre el que el ratón está posicionado.

\subsection{Realizar varias medidas}

Dada por finalizada la lluvia de ideas que añadir como mejoras, había que realizar todo aquello que se decidió implementar más adelante.

Hablo principalmente de la \textbf{gestión de reglas para realizar varias medidas} en cualquiera de los dos visores principales. Para ello había que dotar a la interfaz de un cuadro con las reglas con opciones para añadir, eliminar y habilitar o deshabilitar.

Dado que contaba con la experiencia justa con Qt, se realizó en primer lugar la gestión de elementos de la interfaz para poder posteriormente mapear cada item de las cajas a un \texttt{vtkDistanceWidget} distinto.

\begin{figure}[H]
	\centering
	\includegraphics[width=12cm]{imagenes/varias_reglas}
	\caption{A la derecha, barra de las reglas y en los visores cada una de ellas midiendo algo distinto}
	\label{fig:varias_reglas}
\end{figure}

\subsection{Cambiar color de fondo de los visores}

Una de las primeras ideas que se pensaron fue la de poder permitir cambiar al usuario el \textbf{color de fondo de los \textit{widget}} donde se visualizan tanto el volumen como la malla generada. Pero no se realizó hasta al final pues había cosas más importantes que implementar.

Con esta pequeña mejora junto a otras menores en la interfaz, se dio por terminada la fase de desarrollo del software. Cumpliendo los requisitos iniciales y añadiendo múltiples mejoras que han acabado formando un software bastante completo pero al mismo tiempo sencillo de utilizar.

\subsection{Web de presentación}

Una vez acabado el software y creado un instalador para Windows, aprovechando la funcionalidad de \textit{GitHub Pages} que permite alojar gratuitamente una web para un repositorio, se creó una \textbf{web estática} en la que presentar el software y proporcionar un enlace de descarga, así como enlaces para poder contactar conmigo. La dirección a ésta es: \href{http://fblupi.github.io/3DCurator/}{http://fblupi.github.io/3DCurator/}.
\chapter{Ejemplos}

En este capítulo se presentará el software desde un punto de vista práctico mostrando ejemplos de uso sobre dos esculturas a las que previamente se les hicieron una TC.

Estas esculturas son las de \href{http://patrimonio3d.ugr.es/index.php/granada/escultura/item/18-inmaculada-concepcion}{\textbf{Inmaculada Concepción}} y \href{http://patrimonio3d.ugr.es/index.php/granada/escultura/item/6-san-juan-evangelista}{\textbf{San Juan Evangelista}} (Figura \ref{fig:figuras_reales}), ambas patrimonio de la Universidad de Granada y cuyos datos DICOM han sido proporcionados por el proyecto de Portal Virtual de Patrimonio de las Universidades Andaluzas, coordinado por la Universidad de Granada.

\begin{figure}[H]
	\centering
	\includegraphics[width=7cm]{imagenes/figuras_reales}
	\caption{Esculturas utilizadas para realizar las pruebas. Inmaculada Concepción (izquierda) y San Juan Evangelista (derecha)}
	\label{fig:figuras_reales}
\end{figure}

\section{Presets}

Se han creado cuatro \textit{presets} distintos de funciones de transferencias:

\begin{itemize}
	\item \textbf{CT-OnlyWood}: Para mostrar tan solo la madera.
	\item \textbf{CT-OnlyStucco}: Para mostrar tan solo el estuco.
	\item \textbf{CT-OnlyMetal}: Para mostrar el metal (también se ve con mucha transparencia la madera para tener una referencia de dónde se encuentra el metal).
	\item \textbf{CT-WoodSculpture}: Muestra todos los materiales. Es el utilizado al abrir el programa.
\end{itemize}

Gracias a estos se puede ver que la escultura de Inmaculada Concepción (Figura \ref{fig:inmaculada_concepcion}) no tiene objetos metálicos como pueden ser clavos en su interior. Y además, tiene una capa de estuco bastante gruesa sobre todo en el manto, la cara y las manos.

Al contrario, la escultura de San Juan Evangelista (Figura \ref{fig:san_juan_evangelista}) tiene bastante menos estuco y se pueden observar cinco clavos: dos en los pies, dos en la pierna (bastante torcidos) y uno en la cintura.

\begin{figure}[H]
	\centering
	\includegraphics[width=12cm]{imagenes/inmaculada_concepcion}
	\caption{Reconstrucción volumétrica de Inmaculada Concepción usando los cuatro \textit{presets} proporcionados}
	\label{fig:inmaculada_concepcion}
\end{figure}

\begin{figure}[H]
	\centering
	\includegraphics[width=12cm]{imagenes/san_juan_evangelista}
	\caption{Reconstrucción volumétrica de San Juan Evangelista usando los cuatro \textit{presets} proporcionados}
	\label{fig:san_juan_evangelista}
\end{figure}

Se puede elegir entre estos cuatro \textit{presets} o cambiar la función de transferencia usando las gráficas (Figura \ref{fig:pestana_funcion_de_transferencia}).

\begin{figure}[H]
	\centering
	\includegraphics[width=9cm]{imagenes/pestana_funcion_de_transferencia}
	\caption{Pestaña de la GUI donde se puede modificar la función de transferencia}
	\label{fig:pestana_funcion_de_transferencia}
\end{figure}

Desde la GUI (Figura \ref{fig:pestana_funcion_de_transferencia}) además de tener las gráficas y los botones para cambiar de \textit{preset} se puede importar una función de transferencia o exportar la que se esté usando en ese momento para poder utilizarla en cualquier otro momento o compartirla con otros usuarios.

\section{Plano de corte}

Para interactuar con el plano hay que hacer click derecho sobre éste y moverlo. Para girarlo hay que hacer el click derecho en los bordes de éste. Además desde la GUI (Figura \ref{fig:gui_plano}) se puede cambiar a posiciones por defecto del plano (rojo: sagital, verde: axial y azul: coronal), guardar una imagen del corte y activar o desactivar el plano para que no se muestre en el visor de la reconstrucción volumétrica en 3D.

\begin{figure}[H]
	\centering
	\includegraphics[width=12.5cm]{imagenes/gui_plano}
	\caption{GUI completa con la pestaña del plano activa donde se puede ver el corte que produce el plano oblicuo en la escultura}
	\label{fig:gui_plano}
\end{figure}

La visualización de cortes es tal vez la funcionalidad más útil para un restaurador y es que puede ver el interior de la figura sin tener que dañarla.

En el caso de la Inmaculada Concepción, por ejemplo, en la zona inferior (Figura \ref{fig:corte_inmaculada_concepcion_agujero}) se pueden apreciar hasta cinco piezas de madera distintas: las dos principales base de la escultura,  dos en los laterales y una frontal en la que está la cabeza de un ángel esculpido. Además en este corte se aprecia menos cantidad de estuco que en las zonas donde se encuentra el manto. Se puede observar también el agujero en el centro utilizado para colocar la figura en su soporte, así como los anillos de la madera, útil para determinar la edad de ésta.

\begin{figure}[H]
	\centering
	\includegraphics[width=12.5cm]{imagenes/corte_inmaculada_concepcion_agujero}
	\caption{Corte en la zona inferior de la escultura de Inmaculada Concepción}
	\label{fig:corte_inmaculada_concepcion_agujero}
\end{figure}

Examinando la escultura de San Juan Evangelista, también en la zona inferior (Figura \ref{fig:corte_san_juan_evangelista_clavos_pies}) se pueden observar un par de piezas de madera y los dos clavos de los pies.

\begin{figure}[H]
	\centering
	\includegraphics[width=12.5cm]{imagenes/corte_san_juan_evangelista_clavos_pies}
	\caption{Corte en la zona inferior de la escultura de San Juan Evangelista}
	\label{fig:corte_san_juan_evangelista_clavos_pies}
\end{figure}

En el corte anterior se puede ver algo de distorsión producida por el metal, pero donde mejor se puede observar este fenómeno es en el clavo del centro donde parece que hay una zona hueca y otra rellena de estuco, pero realmente estas zonas tienen estos valores por culpa del ruido que produce el clavo durante el escaneo.

\begin{figure}[H]
	\centering
	\includegraphics[width=12.5cm]{imagenes/corte_san_juan_evangelista_clavo}
	\caption{Corte en la zona central de la escultura de San Juan Evangelista donde se encuentra el clavo}
	\label{fig:corte_san_juan_evangelista_clavo}
\end{figure}

\section{Reglas}

Una vez obtenidos los cortes, es muy útil tener una herramienta para poder hacer medidas. para así conocer el tamaño de las piezas, agujeros, capas de estuco, clavos...

\begin{figure}[H]
	\centering
	\includegraphics[width=12.5cm]{imagenes/medidas_inmaculada_concepcion_agujero}
	\caption{A la izquierda el diámetro y a la derecha la altura del agujero en el que se introduce el soporte donde se coloca la Inmaculada Concepción}
	\label{fig:medidas_inmaculada_concepcion_agujero}
\end{figure}

Por ejemplo se podrían obtener las dimensiones del agujero de la Inmaculada Concepción (Figura \ref{fig:medidas_inmaculada_concepcion_agujero}).

En el caso de el San Juan Evangelista se podrían medir, por ejemplo, los clavos (Figura\ref{fig:medidas_san_juan_evangelista_clavo}).

\begin{figure}[H]
	\centering
	\includegraphics[width=12.5cm]{imagenes/medidas_san_juan_evangelista_clavo}
	\caption{A la izquierda tamaño en profundidad de la pieza y el clavo central y a la derecha tamaño del clavo del pie izquierdo}
	\label{fig:medidas_san_juan_evangelista_clavo}
\end{figure}

\section{Extraer malla}

Para extraer la malla de triángulos el usuario puede elegir entre los \textit{presets} con los valores de isosuperficie de la madera (que incluye estuco y metal), el estuco (que incluye el metal) y el metal. También puede utilizar un valor personalizado desplazando un \textit{slider} (Figura \ref{fig:pestana_malla}).

\begin{figure}[H]
	\centering
	\includegraphics[width=12.5cm]{imagenes/pestana_malla}
	\caption{Pestaña de la GUI para extraer la malla de triángulos}
	\label{fig:pestana_malla}
\end{figure}

Una vez extraída la malla en formato STL se puede usar otro software para trabajar con ella (Figura \ref{fig:malla_clavos_meshlab}).

\begin{figure}[H]
	\centering
	\includegraphics[width=8cm]{imagenes/malla_clavos_meshlab}
	\caption{Malla de los clavos del San Juan Evangelista importada en MeshLab \cite{meshlab}}
	\label{fig:malla_clavos_meshlab}
\end{figure}

\chapter{Conclusiones y trabajos futuros}

En este último capítulo se comentarán las conclusiones a las que se han llegado así como las posibles mejoras en el futuro que se podrían realizar.

\section{Conclusiones}

El uso de la \textbf{TC médica como técnica para examinar las esculturas} de madera policromada en lugar de la radiografía tradicional hemos visto que puede tener muchas ventajas pues puede examinarse el interior de la figura en un espacio 3D sin los problemas de superposición de planos que se daban con la radiografía.

\textbf{Conocer la estructura interna de la escultura es primordial} a la hora de realizar un posterior proceso de conservación y restauración por parte de los restauradores, pero la escasez de herramientas para ello puede ser uno de los motivos por el que esta técnica no ha proliferado todavía.

Durante este proyecto se ha desarrollado un software completo con el que examinar los datos DICOM obtenidos al someter a una escultura a una TC.

Se ha procurado realizar una herramienta \textbf{sencilla de utilizar}, que sea amigable a usuarios que pueden no estar muy habituados al uso de ordenadores para realizar su trabajo. Pero no por ello carente de funcionalidad. Y es que se ha logrado implementar todo lo que se planteó en un principio además de numerosas mejoras.

Por lo que ahora el software cuenta principalmente con la siguiente funcionalidad:

\begin{itemize}
	\item Leer datos DICOM.
	\item Reconstruir volumen a partir de datos DICOM.
	\item Generar y visualizar cortes producidos en la figura por un plano.
	\item Guardar imágenes del volumen y los cortes.
	\item Cambiar de función de transferencia para visualizar unos u otros materiales.
	\item Importar y exportar funciones de transferencia.
	\item Eliminar partes innecesarias del volumen a la hora de la visualización.
	\item Realizar medidas tanto en el volumen como en los cortes.
	\item Generar y exportar una malla de triángulos que contenga un material específico para poder ser posteriormente imprimida.
\end{itemize}

\section{Trabajos futuros}

Las perspectivas futuras del software son altas y pueden realizarse diversas mejoras. En el apartado de nueva funcionalidad:

\begin{itemize}
	\item Además de medir distancias entre dos puntos como se mide ahora se podrían \textbf{medir ángulos} de piezas o definir y \textbf{medir áreas y volúmenes}.
	\item También se podrían \textbf{detectar y definir subvolúmenes de partes} de la escultura con los que trabajar de forma independiente.
	\item Se podría agregar funcionalidad para permitir \textbf{realizar anotaciones} a los restauradores y así no tener que usar lápiz y papel o un software adicional para realizar esta tarea.
	\item Para no tener que escoger la misma carpeta con los datos DICOM cada vez que se importa una escultura, se podría tener una \textbf{base de datos persistente local} donde importar los distintos conjuntos de datos y poder seleccionarlos directamente desde ahí abstrayéndose de dónde están almacenados.
\end{itemize}

También, aprovechando que VTK está disponible para Android (aunque todavía no es estable) se podría migrar la aplicación a \textbf{dispositivos móviles} aprovechando el auge de estos.
%
%\input{capitulos/04_Analisis}
%
%\input{capitulos/05_Diseno}
%
%\input{capitulos/06_Implementacion}
%
%\input{capitulos/07_Pruebas}
%
%\input{capitulos/08_Conclusiones}
%
%%\chapter{Conclusiones y Trabajos Futuros}
%
%
%%\nocite{*}
\bibliography{bibliografia/bibliografia}
\addcontentsline{toc}{chapter}{Bibliografía}
\bibliographystyle{plain}
%
\appendix
\chapter{Manual de usuario}

\section{Requisitos}

\begin{itemize}
	\item Microsoft Windows 7 o superior
	\item Drivers gráficos compatibles con OpenGL 4
	\item 2GB RAM
\end{itemize}

\section{Instalación}

Instalar usando \texttt{3DCurator.msi} siguiendo los siguientes pasos:

\begin{figure}[H]
	\centering
	\includegraphics[width=9cm]{imagenes/instalacion_1}
	\caption{Hacer click en Siguiente}
	\label{fig:instalacion_1}
\end{figure}

\begin{figure}[H]
	\centering
	\includegraphics[width=9cm]{imagenes/instalacion_2}
	\caption{Hacer click en Siguiente o cambiar cualquiera de las opciones}
	\label{fig:instalacion_2}
\end{figure}

\begin{figure}[H]
	\centering
	\includegraphics[width=9cm]{imagenes/instalacion_3}
	\caption{Hacer click en Siguiente, dar permisos de administrador y comenzará a instalarse}
	\label{fig:instalacion_3}
\end{figure}

\begin{figure}[H]
	\centering
	\includegraphics[width=9cm]{imagenes/instalacion_4}
	\caption{Hacer click en Cerrar. Al instalar se habrá creado un acceso directo en el escritorio y en el menú de inicio}
	\label{fig:instalacion_4}
\end{figure}

\section{Capturas de pantalla}

\begin{figure}[H]
	\centering
	\includegraphics[width=12.5cm]{imagenes/gui_1}
	\caption{Captura de la GUI (Pestaña \textit{Cortes})}
	\label{fig:gui_1}
\end{figure}

\begin{figure}[H]
	\centering
	\includegraphics[width=12.5cm]{imagenes/gui_2}
	\caption{Captura de la GUI (Pestaña \textit{Función de transferencia})}
	\label{fig:gui_2}
\end{figure}

\begin{figure}[H]
	\centering
	\includegraphics[width=12.5cm]{imagenes/gui_3}
	\caption{Captura de la GUI (Pestaña \textit{Extraer malla})}
	\label{fig:gui_3}
\end{figure}

\begin{figure}[H]
	\centering
	\includegraphics[width=12.5cm]{imagenes/gui_4}
	\caption{Captura de la GUI (Pestaña \textit{Propiedades})}
	\label{fig:gui_4}
\end{figure}

\section{Instrucciones de uso}

\subsection{Abrir directorio DICOM y visualizar datos}

Botón 1, Archivo $ \rangle $ Abrir... o \texttt{Ctrl+O}. Aparecerá una ventana donde se podrá elegir un directorio del sistema. Seleccionar aquel que tenga los archivos DICOM que se quieran importar y pulsar en abrir. Entonces empezará a cargar datos (puede tardar unos segundos). Si hay algún archivo en el directorio que no sea DICOM se mostrará un mensaje de error, pero si ha conseguido cargar el resto se puede continuar con la ejecución del programa sin problema.

Al abrir el archivo DICOM se mostrará la reconstrucción volumétrica en el visor principal, el corte producido con el plano en el visor del plano y la malla en el visor de la malla.

Si se mueve el ratón por el visor del corte, abajo de este aparecerá el valor de densidad del pixel sobre el que está el ratón.

\subsubsection{Visor de volumen y malla}

\begin{itemize}
	\item \textbf{Girar cámara}: Click izdo. + arrastrar
	\item \textbf{Mover cámara}: Click central + arrastrar
	\item \textbf{Zoom}: Click dcho. + arrastrar o rueda del ratón
\end{itemize}

\subsubsection{Visor del corte}

\begin{itemize}
	\item \textbf{Mover}: Click central + arrastrar
	\item \textbf{Zoom}: Click dcho. + arrastrar o rueda del ratón
\end{itemize}

\subsection{Eliminar partes de la figura}

Botón 2, Editar $ \rangle $ Borrar partes o \texttt{Ctrl+May+D}. Se cambiará el visor del volumen de color y no se podrá girar la cámara.

Para borrar hacer click en un punto de la isla que se desea borrar. Entendiendo como isla parte que está separada de las demás. El proceso tardará unos segundos dependiendo del tamaño de la parte que se va a borrar. Al borrar se podrá ver el efecto del borrado y confirmar o volver a como estaba antes de borrar.

En ocasiones se necesitará hacer más de un borrado para borrar una isla por completo.

\subsection{Cambiar plano}

\subsubsection{Mostrar/Esconder}

En la pestaña \textit{Cortes} Botón 10, Editar $ \rangle $ Mostrar/Esconder plano o \texttt{Ctrl+May+H}. Esconderá o mostrará el plano en el visor de volumen. No se actualizará la imagen del corte si se ha escondido hasta que no se vuelva a mostrar.

\subsubsection{Posiciones por defecto}

Para colocarlas centradas en un plano anatómico:

\begin{itemize}
	\item \textbf{Sagital}: En la pestaña \textit{Cortes} Botón 11, Editar $ \rangle $ Plano sagital o \texttt{Ctrl+May+S}
	\item \textbf{Axial}: En la pestaña \textit{Cortes} Botón 12, Editar $ \rangle $ Plano axial o \texttt{Ctrl+May+A}
	\item \textbf{Coronal}: En la pestaña \textit{Cortes} Botón 13, Editar $ \rangle $ Plano coronal o \texttt{Ctrl+May+C}
\end{itemize}

\subsubsection{Mover}

El plano se mueve haciendo click derecho sobre éste y arrastrando hacia donde se desea. Para girarlo hacer el click en los extremos del plano.

\subsection{Guardar imagen del volumen}

Botón 3, Archivo $ \rangle $ Exportar figura... o \texttt{Ctrl+F}. Aparecerá una ventana donde se elegirá dónde, con qué nombre y con qué formato guardar la imagen.

\subsection{Guardar imagen del corte}

En la pestaña \textit{Cortes} Botón 14, Archivo $ \rangle $ Exportar corte... o \texttt{Ctrl+F}. Aparecerá una ventana donde se elegirá dónde, con qué nombre y con qué formato guardar la imagen.
 
\subsection{Cambiar color de fondo de visores}

En la pestaña \textit{Propiedades} pulsar sobre el botón coloreado a la derecha del nombre del visor cuyo color de fondo quiera ser cambiado. Aparecerá una ventana para elegir el color.

Si se desean restaurar los colores por defecto, pulsar en el Botón 29.

\subsection{Cambiar material del volumen}

En la pestaña \textit{Propiedades} cambiar los valores de las distintas componentes del material y pulsar en el Botón 28.

Si se desea restaurar el material por defecto, pulsar en el Botón 27.

\subsection{Función de transferencia}

\subsubsection{Cambiar \textit{preset}}

\begin{itemize}
	\item \textbf{Completo}: En la pestaña \textit{Función de transferencia} Botón 11, Editar $ \rangle $ Preset completo o \texttt{F1}
	\item \textbf{Madera}: En la pestaña \textit{Función de transferencia} Botón 12, Editar $ \rangle $ Preset madera o \texttt{F2}
	\item \textbf{Estuco}: En la pestaña \textit{Función de transferencia} Botón 13, Editar $ \rangle $ Preset estuco o \texttt{F3}
	\item \textbf{Metal}: En la pestaña \textit{Función de transferencia} Botón 14, Editar $ \rangle $ Preset metal o \texttt{F4}
\end{itemize}

\subsubsection{Importar}

En la pestaña \textit{Función de transferencia} Botón 15, Herramientas $ \rangle $ Importar preset... o \texttt{Ctrl+May+I}. Aparecerá una ventana donde se podrá escoger el archivo XML con el \textit{preset} que se quiere importar.

En la pestaña \textit{Propiedades} cambiar los valores de las distintas componentes del material y pulsar en el Botón 28.

Si se desea restaurar el material por defecto, pulsar en el Botón 27.

\subsubsection{Exportar}

Editar nombre y descripción en los campos dedicados para ello y, en la pestaña \textit{Función de transferencia} Botón 16, Herramientas $ \rangle $ Exportar preset... o \texttt{Ctrl+May+E}. Aparecerá una ventana donde se podrá escoger el nombre y la ubicación del archivo con el \textit{preset} que se exportará usando la función de transferencia actual.

\subsubsection{Editar}
 
Interactuar con las tres gráficas de la Caja 21 en la pestaña \textit{Función de transferencia}. Se puede cambiar el rango máximo y mínimo que aparece en la gráfica con los \textit{sliders} inferiores. Para cambiar los puntos: 

\begin{itemize}
	\item \textbf{Añadir punto}: Click
	\item \textbf{Seleccionar punto}: Click sobre el punto
	\item \textbf{Eliminar punto}: Click central sobre el punto o seleccionar y \texttt{Del} o \texttt{Supr}.
	\item \textbf{Mover punto}: Seleccionar punto y arrastrar
	\item \textbf{Cambiar color}: (Solo en la de color) Doble click. Aparecerá una ventana de selección de color
\end{itemize}

\subsection{Generar malla}

En la pestaña \textit{Extraer malla} usar el Slider 25 para cambiar el valor de isosuperficie. Se generará en unos segundos la malla.

Para elegir entre cualquiera de los \textit{presets}:

\begin{itemize}
	\item \textbf{Madera}: En la pestaña \textit{Extraer malla} Botón 22, Editar $ \rangle $ Malla madera. Incluye los materiales madera, estuco y metal.
	\item \textbf{Estuco}: En la pestaña \textit{Extraer malla} Botón 23, Editar $ \rangle $ Malla estuco. Incluye los materiales estuco y metal.
	\item \textbf{Metal}: En la pestaña \textit{Extraer malla} Botón 24, Editar $ \rangle $ Malla metal. Incluye el material metal.
\end{itemize}

\subsection{Extraer malla}

En la pestaña \textit{Extraer malla} Botón 26, Herramientas $ \rangle $ Extraer malla... o \texttt{Ctrl+May+M}. Aparecerá una ventana donde se elegirá el nombre y la ubicación de la malla que se exportará en formato STL.

\subsection{Realizar medidas}

\subsubsection{Añadir regla}

Según el visor donde se realizará la medida pulsar el botón 6 (visor de volumen) o 9 (visor de cortes). Se hará click derecho en el punto inicial y en el final y aparecerá la medida en milímetros.

\subsubsection{Eliminar regla}

Seleccionar una regla de cualquiera de las cajas de reglas y pulsar en el botón 5 (reglas de volumen) u 8 (reglas de cortes). La regla desaparecerá.

\subsubsection{Mostrar/Esconder regla}

Seleccionar una regla de cualquiera de las cajas de reglas y pulsar en el botón 4 (reglas de volumen) u 7 (reglas de cortes). La regla se mostrará o se esconderá según su estado previo.

\subsubsection{Mover regla}

Click izquierdo en cualquiera de los puntos inicial o final de una regla y arrastrar el ratón.
%%\input{apendices/paper/paper}
%\input{glosario/entradas_glosario}
% \addcontentsline{toc}{chapter}{Glosario}
% \printglossary
%\chapter*{}
%\thispagestyle{empty}

\end{document}
