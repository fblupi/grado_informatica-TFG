\chapter{Especificación de requisitos}

Este capítulo es una Especificación de Requisitos Software (ERS) para el software que se va a realizar siguiendo las directrices dadas por el estándar IEEE830 \cite{iee830}.

\section{Introducción}

\subsection{Propósito}

Este capítulo de especificación de requisitos tiene como objetivo definir las especificaciones funcionales y no funcionales para el desarrollo de un software que permitirá visualizar e interactuar con los datos DICOM obtenidos al someter a una escultura a un TAC. Éste será utilizado principalmente por restauradores.

\subsection{Ámbito del sistema}

En la actualidad los datos DICOM obtenidos tras un TAC se utilizan, principalmente, en el campo donde surgieron, la medicina. No obstante, esto no significa que solo se pueda aplicar ahí. Con este software, llamadao \myTitle, se tratará de trasladar esta técnica al campo de la restauración de bienes culturales y poder visualizar e interactuar con los datos DICOM obtenidos con esculturas.

\subsection{Definiciones, acrónimos y abreviaturas}

\begin{itemize}
	\item \textbf{ERS}: Especificación de Requisitos Software
	\item \textbf{Lele}: Lolo
\end{itemize}

\subsection{Visión general del documento}

Este capítulo consta de tres secciones:
\begin{itemize}
	\item En la primera sección se realiza una introducción a éste y se proporciona una visión general de la ERS.
	\item En la segunda sección se realiza una descripción general a alto nivel del software, describiendo los factores que afectan al producto y a sus requisitos y con el objetivo de conocer las principales funcionalidades de éste.
	\item En la tercera sección se definen detalladamente los requisitos que deberá satisfacer el software.
\end{itemize}

\section{Descripción general}

\subsection{Perspectiva del producto}

\subsection{Funciones del producto}

\subsection{Características de los usuarios}

\subsection{Restricciones}

\subsection{Suposiciones y dependencias}

\subsection{Requisitos futuros}

\section{Requisitos específicos}

\subsection{Interfaces externas}

\subsection{Funciones}

\subsection{Requisitos de rendimiento}

\subsection{Restricciones de diseño}

\subsection{Atributos del sistema}

\subsection{Otros requisitos}

\section{Apéndices}